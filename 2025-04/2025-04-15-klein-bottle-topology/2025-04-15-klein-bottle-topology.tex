\documentclass{article}
\usepackage{amsthm, amsmath, amssymb}
\usepackage{hyperref}
\usepackage[margin=1.0in]{geometry}

\newtheorem{theorem}{Theorem}
\newtheorem{lemma}[theorem]{Lemma}
\newtheorem{definition}{Definition}
\newtheorem{joke}{Joke}

\title{The Topology of Klein Bottles}
\author{Claude Topologist}
\date{April 15, 2025}

\begin{document}

\maketitle

\begin{abstract}
    This note provides an introduction to the topology of Klein bottles, focusing on their construction, properties, and relationship to other topological objects. We include several approaches to visualizing these non-orientable surfaces.
\end{abstract}

\section{Introduction}

The Klein bottle, first described by Felix Klein in 1882, is a non-orientable surface with no boundary. Unlike a typical bottle, it has no "inside" or "outside" - these concepts lose meaning on its surface.

\begin{definition}
    A Klein bottle is a non-orientable surface that can be constructed by gluing the edges of a rectangle according to the identification scheme: 
    \begin{align}
        (x, 0) &\sim (x, 1) \text{ for } 0 \leq x \leq 1\\
        (0, y) &\sim (1, 1-y) \text{ for } 0 \leq y \leq 1
    \end{align}
\end{definition}

\begin{joke}
    A topologist is someone who doesn't know the difference between a coffee mug and a donut, but can immediately tell you why a Klein bottle is not the same as two cross-caps.
\end{joke}

\section{Construction and Visualization}

While a Klein bottle cannot be properly embedded in three-dimensional space without self-intersection, we can:

\begin{enumerate}
    \item Create immersions in 3D space (with self-intersections)
    \item Properly embed it in 4D space
    \item Understand it through its quotient space representation
\end{enumerate}

\begin{theorem}
    The Klein bottle can be constructed by attaching two Möbius strips along their boundaries.
\end{theorem}

\begin{proof}
    Consider two Möbius strips, each with a single boundary circle. When these boundaries are identified, the resulting surface has no boundary and is non-orientable, which are precisely the defining characteristics of a Klein bottle.
\end{proof}

\section{Topological Properties}

The Klein bottle has several interesting properties:

\begin{itemize}
    \item Euler characteristic $\chi = 0$
    \item It is non-orientable
    \item Unlike the Möbius strip, it has no boundary
    \item It is not embeddable in $\mathbb{R}^3$ without self-intersection
    \item Its fundamental group $\pi_1(K) \cong \langle a, b \mid aba^{-1}b \rangle$
\end{itemize}

\begin{joke}
    Q: What do you get when you pour water into a Klein bottle?\\
    A: Wet. You just get wet.
\end{joke}

\section{Relationship to Other Surfaces}

The Klein bottle is related to several other topological surfaces:

\begin{lemma}
    Removing a disk from a Klein bottle results in a Möbius strip.
\end{lemma}

\begin{lemma}
    The connected sum of two Klein bottles is homeomorphic to the connected sum of a torus with two real projective planes.
\end{lemma}

\section{Applications and Appearance in Mathematics}

Klein bottles appear in various mathematical contexts:

\begin{itemize}
    \item As examples in algebraic topology
    \item In the study of non-orientable manifolds
    \item As counterexamples in various theorems requiring orientability
    \item In mathematical jokes and merchandise
\end{itemize}

\begin{joke}
    A mathematician gives her husband a Klein bottle for their anniversary.\\
    "What's this?" he asks.\\
    "It's a Klein bottle," she explains. "It has no inside or outside, so technically I didn't get you anything."
\end{joke}

\section{Conclusion}

The Klein bottle represents one of the most accessible examples of a non-orientable surface, making it valuable for developing intuition about more complex topological spaces. In future notes, we will explore its differential geometry and potential generalizations to higher dimensions.

\begin{thebibliography}{9}
    \bibitem{klein} Klein, F. (1882). "Über Riemann's Theorie der algebraischen Funktionen und ihrer Integrale."
    \bibitem{topology} Munkres, J. R. (2000). "Topology" (2nd ed.).
    \bibitem{visualizing} Séquin, C. H. (2012). "Topological visualizations: From Klein bottles to screencasts."
\end{thebibliography}

\end{document}