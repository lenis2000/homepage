\documentclass[letterpaper,11pt,oneside,reqno]{article}

%%%%%%%%%%%%%%%%%%%%%%%%%%%%%%%%%%%%%%%%%%%%%%%%%%%%%%%%%%%%

\usepackage[pdftex,backref=page,colorlinks=true,linkcolor=blue,citecolor=red]{hyperref}
\usepackage[alphabetic,nobysame]{amsrefs}

%%%%%%%%%%%%%%%%%%%%%%%%%%%%%%%%%%%%%%%%%%%%%%%%%%%%%%%%%%%%
%main packages
\usepackage{amsmath,amssymb,amsthm,amsfonts,mathtools}
\usepackage{graphicx,color}
\usepackage{upgreek}
\usepackage[mathscr]{euscript}

%equations
\allowdisplaybreaks
\numberwithin{equation}{section}

%tikz
\usepackage{tikz}
\usetikzlibrary{shapes,arrows,positioning,decorations.markings}

%conveniences
\usepackage{array}
\usepackage{adjustbox}
\usepackage{cleveref}
\usepackage{enumerate}

%paper geometry
\usepackage[DIV=12]{typearea}

%%%%%%%%%%%%%%%%%%%%%%%%%%%%%%%%%%%%%%%%%%%%%%%%%%%%%%%%%%%%
%draft-specific
\synctex=1
% \usepackage{refcheck,comment}

%%%%%%%%%%%%%%%%%%%%%%%%%%%%%%%%%%%%%%%%%%%%%%%%%%%%%%%%%%%%
%this paper specific
\newcommand{\ssp}{\hspace{1pt}}

%%%%%%%%%%%%%%%%%%%%%%%%%%%%%%%%%%%%%%%%%%%%%%%%%%%%%%%%%%%%
\newtheorem{proposition}{Proposition}[section]
\newtheorem{lemma}[proposition]{Lemma}
\newtheorem{corollary}[proposition]{Corollary}
\newtheorem{theorem}[proposition]{Theorem}
%%%%%%%%%%%%%%%%%%%%%%%%%%%%%%%%%%%%%%%%%%%%%%%%%%%%%%%%%%%%
\theoremstyle{definition}
\newtheorem{definition}[proposition]{Definition}
\newtheorem{remark}[proposition]{Remark}
\newtheorem{exercise}[proposition]{Exercise}
%%%%%%%%%%%%%%%%%%%%%%%%%%%%%%%%%%%%%%%%%%%%%%%%%%%%%%%%%%%%

\begin{document}
\title{Random Fibonacci Words}

% OTHER AUTHORS

\author{Leonid Petrov; based on the joint work with Jeanne Scott \cite{PetrovScott2024Fibonacci}}

\date{Talk at University of Illinois at Urbana-Champaign, Integrability \& Representation Theory (IRT) Seminar, February 20, 2025}


\maketitle

\begin{abstract}
	Fibonacci words are words of 1's and 2's, graded by the total sum of the digits. They form a differential poset ($\mathbb{YF}$) which is an estranged cousin of the Young lattice powering irreducible representations of the symmetric group. We introduce families of "coherent" measures on $\mathbb{YF}$ depending on many parameters, which come from the theory of clone Schur functions \cite{okada1994algebras}. We characterize parameter sequences ensuring positivity of the measures, and we describe the large-scale behavior of some ensembles of random Fibonacci words. The subject has connections to total positivity of tridiagonal matrices, Stieltjes moment sequences, orthogonal polynomials from the (q-)Askey scheme, and residual allocation (stick-breaking) models.
\end{abstract}

\section{Motivation 1. De Finetti's theorem and Pascal triangle}

\subsection{}

\begin{definition}
	A sequence $X_1,X_2,\ldots $
	of binary random variables 
	(taking values in $\{0,1\}$)
	is called \emph{exchangeable}
	if for any $n$ and any permutation $\sigma$ of $\{1,2,\ldots,n\}$ the joint distribution of $X_1,X_2,\ldots,X_n$ is the same as the joint distribution of $X_{\sigma(1)},X_{\sigma(2)},\ldots,X_{\sigma(n)}$.
\end{definition}

Exchangeable sequences are more than just Bernoulli iid sequences with some
parameter $p\in[0,1]$. Consider the Polya urn scheme.

Start with an urn containing $b$ black and $w$ white balls.
At each step, draw a ball uniformly at random from the urn
and put it back along with another ball of the same color.

\begin{exercise}
	The sequence of ball colors drawn from the urn is exchangeable.
\end{exercise}

At time $n$, there are $n$ new balls in the urn, and the distribution of the number of, say, 
black balls, 
\begin{equation*}
	\operatorname{\mathbb{P}}\left( \textnormal{black}=k \right) = M_n(k),
	\qquad k=0,1,\ldots,n,
\end{equation*}
is called the ($n$-th) \emph{coherent measure}.
We can talk about $S_n$, the random variable which is the number of black balls drawn by time $n$.
The coherent measures $M_n$ for various $n$ satisfy certain linear 
recurrence relations.

One can convince oneself that the space of coherent measures is 
the same as the space of exchangeable random sequences of 0's and 1's.
This space is a convex set, moreover, it is a simplex.

Extreme points of this simplex are given by iid sequences, 
that is, Bernoulli product measures on $\left\{ 0,1 \right\}^\infty$.
This is de Finetti's theorem.

\subsection{}

Coherent measures on Pascal triangle are related to exchangeable
sequences of 0's and 1's. The \emph{boundary} of the Pascal
triangle encodes all possible coherent measures via the law of large numbers,
\begin{equation*}
	\frac{S_n}{n}\to \mu \quad \textnormal{on} \quad [0,1].
\end{equation*}
Extreme measures correspond to delta point masses.
For example, the Polya urn for $a=b=1$ corresponds to $\mu$ being the
uniform measure on $[0,1]$.

\subsection{Lonely paths}

There are two distinguished paths in the Pascal triangle, the 
\emph{lonely paths} $0\to00\to000\to\ldots $
and 
$1\to 11\to111\to\ldots $, which are characterized by the 
property that \cite{KerovGoodman1997}
\begin{quote}
	All but finitely many vertices in the path have a single immediate predecessor.
\end{quote}
These paths correspond to the extreme measures 
with $\mu=\delta_0$ and $\mu=\delta_1$, respectively.


\section{Motivation 2. Young lattice}

The Young lattice $\mathbb{Y}$ of integer partitions ordered
by the relation ``adding a box''
encodes another meaningful structure --- irreducible representations of the symmetric groups.
The boundary encodes the irreducible representations of the infinite symmetric group $S(\infty)$.

\subsection{}
The Young lattice is a \emph{differential poset} \cite{stanley1988differential},
\cite{fomin1994duality}, 
in the sense that 
\begin{quote}
	for each $\lambda$, there is one more element in the set
	$\left\{ \nu\colon\nu=\lambda+\square \right\}$ than
	in the set $\left\{ \mu\colon\mu=\lambda-\square \right\}$. 
\end{quote}
Differential poset 
property implies that 
for $f^\lambda$ the number of paths from $\varnothing$ to $\lambda$, we have
\begin{equation*}
	\sum_{|\lambda|=n}(f^\lambda)^2=n!, \qquad \textnormal{define}\qquad
	M_n(\lambda)\coloneqq\frac{(f^\lambda)^2}{n!}.
\end{equation*}
The measure $M_n$ is called \emph{Plancherel}, it is coherent
and extremal. It corresponds to the regular representation of $S(\infty)$,
which is irreducible.

\subsection{}

There are two lonely paths here, as well --- corresponding to growing 
one-row and one-column partitions.

\subsection{}

All extreme coherent measures on the Young lattice are given by specializations
of Schur symmetric functions, and have the form
\begin{equation*}
	M_n(\lambda)=s_\lambda(\vec \alpha;\vec \beta;\gamma)\cdot f^\lambda.
\end{equation*}
The problem of describing the boundary of $\mathbb{Y}$ is equivalent to
the problem of finding parameters $\vec \alpha,\vec \beta,\gamma$ such that
the Schur functions $s_\lambda(\vec \alpha;\vec \beta;\gamma)$ are nonnegative
for all $\lambda$.

Schur functions are (essentially) determinants, and 
for the Young lattice, we have a great match between these multiparameter functions and 
extreme coherent measures.


\section{Another differential poset --- the Young--Fibonacci lattice}

\subsection{}

Are there any other differential posets?

























\bibliographystyle{alpha}
\bibliography{bib}

\end{document}
