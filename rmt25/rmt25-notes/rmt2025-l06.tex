\documentclass[letterpaper,11pt,oneside,reqno]{article}

%%%%%%%%%%%%%%%%%%%%%%%%%%%%%%%%%%%%%%%%%%%%%%%%%%%%%%%%%%%%

\usepackage[pdftex,backref=page,colorlinks=true,linkcolor=blue,citecolor=red]{hyperref}
\usepackage[alphabetic,nobysame]{amsrefs}

%%%%%%%%%%%%%%%%%%%%%%%%%%%%%%%%%%%%%%%%%%%%%%%%%%%%%%%%%%%%
%main packages
\usepackage{amsmath,amssymb,amsthm,amsfonts,mathtools}
\usepackage{graphicx,color}
\usepackage{upgreek}
\usepackage[mathscr]{euscript}

%equations
\allowdisplaybreaks
\numberwithin{equation}{section}

%tikz
\usepackage{tikz}
\usetikzlibrary{shapes,arrows,positioning,decorations.markings}

%conveniences
\usepackage{array}
\usepackage{adjustbox}
\usepackage{cleveref}
\usepackage{enumerate}
\usepackage{datetime}

%paper geometry
\usepackage[DIV=12]{typearea}

%%%%%%%%%%%%%%%%%%%%%%%%%%%%%%%%%%%%%%%%%%%%%%%%%%%%%%%%%%%%
%draft-specific
\synctex=1
% \usepackage{refcheck,comment}

%%%%%%%%%%%%%%%%%%%%%%%%%%%%%%%%%%%%%%%%%%%%%%%%%%%%%%%%%%%%
%this paper specific
\newcommand{\ssp}{\hspace{1pt}}

%%%%%%%%%%%%%%%%%%%%%%%%%%%%%%%%%%%%%%%%%%%%%%%%%%%%%%%%%%%%
\newtheorem{proposition}{Proposition}[section]
\newtheorem{lemma}[proposition]{Lemma}
\newtheorem{corollary}[proposition]{Corollary}
\newtheorem{theorem}[proposition]{Theorem}
%%%%%%%%%%%%%%%%%%%%%%%%%%%%%%%%%%%%%%%%%%%%%%%%%%%%%%%%%%%%
\theoremstyle{definition}
\newtheorem{definition}[proposition]{Definition}
\newtheorem{remark}[proposition]{Remark}
%%%%%%%%%%%%%%%%%%%%%%%%%%%%%%%%%%%%%%%%%%%%%%%%%%%%%%%%%%%%

\begin{document}
\title{Lectures on Random Matrices
(Spring 2025)
\\Lecture 6: Steepest descent and local statistics}

\date{February 12, 2025\footnote{\href{https://lpetrov.cc/rmt25/}{\texttt{Course webpage}}
$\bullet$ \href{https://lpetrov.cc/simulations/model/random-matrices/}{\texttt{Live simulations}}
$\bullet$ \href{https://lpetrov.cc/rmt25/rmt25-notes/rmt2025-l06.tex}{\texttt{TeX Source}}
$\bullet$
Updated at \currenttime, \today}}



\author{Leonid Petrov}


\maketitle


\section{Recap}

\subsection{Determinantal structure of the GUE}

Last time, we proved the following result:
\begin{theorem}
\label{thm:determinantal_GUE}
\end{theorem}













%%%%%%%%%%%%%%%%%%%%%%%%%%%%%%%%%%%%%%%%%%%%%%%%%%%%%%%%%%%%
\section{Double Contour Integral Representation for the GUE Kernel}
\label{sec:double-contour}

\subsection{One contour integral representation for Hermite polynomials}

Recall that the GUE kernel is defined by
\[
K_N(x,y)=\sum_{n=0}^{N-1}\psi_n(x)\psi_n(y),
\]
with the orthonormal functions
\[
\psi_n(x)=\frac{1}{\sqrt{h_n}}\,p_n(x)\,e^{-x^2/4},
\]
where the (monic, probabilists') Hermite polynomials are given by
\[
p_n(x)=(-1)^n e^{x^2/2}\frac{d^n}{dx^n}\,e^{-x^2/2},
\]
and satisfy the generating function
\[
\exp\Bigl(xt-\frac{t^2}{2}\Bigr)=\sum_{n\ge0}p_n(x)\frac{t^n}{n!}.
\]

By Cauchy's integral formula we can write
\[
p_n(x)=\frac{n!}{2\pi i}\oint_C\frac{\exp\Bigl(xt-\frac{t^2}{2}\Bigr)}{t^{n+1}}\,dt,
\]
where the contour \(C\) is a simple closed curve encircling the origin.
Therefore,
\begin{equation*}
	\psi_n(x)=\frac{1}{\sqrt{h_n}}\,p_n(x)\,e^{-x^2/4}=
	\frac{e^{-x^2/4}}{\sqrt{h_n}}\frac{n!}{2\pi i}\oint_C\frac{\exp\Bigl(xt-\frac{t^2}{2}\Bigr)}{t^{n+1}}\,dt.
\end{equation*}


\subsection{Another contour integral representation for Hermite polynomials}

Note also that
\[
\int_{-\infty}^{\infty} e^{-t^2+\sqrt{2} i\,t\,x}\,dt
=\sqrt{\pi}\,e^{-x^2/2}.
\]
Differentiating both sides \(n\) times with respect to \(x\) (and using the fact that in our convention the Gaussian appears with \(x^2/2\)) yields
\[
\frac{d^n}{dx^n}\Bigl(e^{-x^2/2}\Bigr)
=\frac{1}{\sqrt{\pi}}\int_{-\infty}^{\infty} \Bigl(\sqrt{2}i\,t\Bigr)^n
e^{-t^2+\sqrt{2} i\,t\,x}\,dt.
\]
Changing variables via \(s=i\,t\) (so that \(t=-i\,s\) and \(dt=-i\,ds\)) one obtains
\[
\frac{d^n}{dx^n}\Bigl(e^{-x^2/2}\Bigr)
=\frac{(\sqrt{2})^n}{i\sqrt{\pi}}
\int_{-i\infty}^{i\infty} s^n\,e^{s^2+\sqrt{2}\,s\,x}\,ds.
\]
Multiplying by \((-1)^n e^{x^2/2}\) we deduce that
\begin{equation}
\label{eq:pn-contour}
p_n(x)
=(-1)^n e^{x^2/2}\frac{d^n}{dx^n}\Bigl(e^{-x^2/2}\Bigr)
=\frac{i\,(\sqrt{2})^n\,e^{x^2/2}}{\sqrt{\pi}}
\int_{-i\infty}^{i\infty} s^n\,e^{s^2-\sqrt{2}\,s\,x}\,ds.
\end{equation}

Now, recall that the orthonormal functions are defined as
\[
\psi_n(x)=\frac{1}{\sqrt{h_n}}\,p_n(x)\,e^{-x^2/4},
\]
so that by \eqref{eq:pn-contour}
\[
\psi_n(x)
=\frac{i\,e^{x^2/4}}{\sqrt{\pi\,h_n}}
\int_{-i\infty}^{i\infty}(\sqrt2 s)^n\,e^{s^2-\sqrt{2}\,s\,x}\,ds
=
\frac{i\,e^{x^2/4}}{\sqrt{2\pi\,h_n}}
\int_{-i\infty}^{i\infty}s^n\,e^{s^2/2-s\,x}\,ds.
\]

\subsection{Double contour integral representation for the GUE kernel}

We have (Problem~\ref{prob:norm})
\begin{equation*}
	h_n=\int_{-\infty}^{\infty} p_n(x)^2\,e^{-x^2/2}\,dx=n!\sqrt{2\pi}.
\end{equation*}
Therefore, we can sum up the kernel (another proof of the Christoffel--Darboux formula):
\begin{align*}
	K_n(x,y)&=
	\sum_{k=0}^{n-1}\psi_k(x)\psi_k(y)
	\\&=
	\sum_{k=0}^{n-1}
	\frac{e^{-x^2/4}}{\sqrt{h_k}}\frac{k!}{2\pi i}\oint_C\frac{\exp\Bigl(xt-\frac{t^2}{2}\Bigr)}{t^{k+1}}\,dt
	\frac{i\,e^{y^2/4}}{\sqrt{2\pi\,h_k}}
	\int_{-i\infty}^{i\infty}s^k\,e^{s^2/2-s\,y}\,ds
	\\&=
	e^{(y^2-x^2)/4}
	\sum_{k=0}^{n-1}
	\frac{1}{4\pi^2}\oint_C\frac{\exp\Bigl(xt-\frac{t^2}{2}\Bigr)}{t^{k+1}}\,dt
	\int_{-i\infty}^{i\infty}s^k\,e^{s^2/2-s\,y}\,ds.
\end{align*}
We can now extend the sum to
$k=-\infty$, and get a formula for the
GUE kernel as a double contour integral:
\begin{equation*}
	K_n(x,y)=\frac{e^{(y^2-x^2)/4}}{4\pi^2}
	\oint_C
	\int_{-\mathbf{i}\infty}^{\mathbf{i}\infty}
	\frac{\exp\left\{ \frac{s^2}{2}-sy-\frac{t^2}{2}+tx \right\}}{s-t}\left( \frac{s}{t} \right)^n\ssp ds\ssp dt.
\end{equation*}
Details will be in the next
\href{https://lpetrov.cc/rmt25/rmt25-notes/rmt2025-l06.pdf}{Lecture 6}.

\begin{remark}
Many other versions of the GUE / unitary invariant ensembles admit determinantal structure:
\begin{enumerate}
	\item The GUE corners process \cite{johansson2006eigenvalues}
    \item The Dyson Brownian motion (e.g., add a GUE to a diagonal matrix)
			\cite{nagao1998multilevel}
		\item GUE corners plus a fixed matrix \cite{Ferrari2014PerturbedGUE}
    \item Corners invariant ensembles with fixed eigenvalues $UDU^\dagger$, where $D$ is a fixed diagonal matrix and $U$ is Haar distributed
			on the unitary group \cite{Metcalfe2011GT}
	\end{enumerate}
\end{remark}


%%%%%%%%%%%%%%%%%%%%%%%%%%%%%%%%%%%%%%%%%%%%%%%%%%%%%%%%%%%%





















\appendix
\setcounter{section}{5}

\section{Problems (due DATE)}





\bibliographystyle{alpha}
\bibliography{bib}


\medskip

\textsc{L. Petrov, University of Virginia, Department of Mathematics, 141 Cabell Drive, Kerchof Hall, P.O. Box 400137, Charlottesville, VA 22904, USA}

E-mail: \texttt{lenia.petrov@gmail.com}


\end{document}
