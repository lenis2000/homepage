\documentclass[letterpaper,11pt,oneside,reqno]{article}

%%%%%%%%%%%%%%%%%%%%%%%%%%%%%%%%%%%%%%%%%%%%%%%%%%%%%%%%%%%%

\usepackage[pdftex,backref=page,colorlinks=true,linkcolor=blue,citecolor=red]{hyperref}
\usepackage[alphabetic,nobysame]{amsrefs}

%%%%%%%%%%%%%%%%%%%%%%%%%%%%%%%%%%%%%%%%%%%%%%%%%%%%%%%%%%%%
%main packages
\usepackage{amsmath,amssymb,amsthm,amsfonts,mathtools}
\usepackage{graphicx,color}
\usepackage{upgreek}
\usepackage[mathscr]{euscript}

%equations
\allowdisplaybreaks
\numberwithin{equation}{section}

%tikz
\usepackage{tikz}
\usetikzlibrary{shapes,arrows,positioning,decorations.markings}

%conveniences
\usepackage{array}
\usepackage{adjustbox}
\usepackage{cleveref}
\usepackage{enumerate}
\usepackage{datetime}

%paper geometry
\usepackage[DIV=12]{typearea}

%%%%%%%%%%%%%%%%%%%%%%%%%%%%%%%%%%%%%%%%%%%%%%%%%%%%%%%%%%%%
%draft-specific
\synctex=1
% \usepackage{refcheck,comment}

%%%%%%%%%%%%%%%%%%%%%%%%%%%%%%%%%%%%%%%%%%%%%%%%%%%%%%%%%%%%
%this paper specific
\newcommand{\ssp}{\hspace{1pt}}

%%%%%%%%%%%%%%%%%%%%%%%%%%%%%%%%%%%%%%%%%%%%%%%%%%%%%%%%%%%%
\newtheorem{proposition}{Proposition}[section]
\newtheorem{lemma}[proposition]{Lemma}
\newtheorem{corollary}[proposition]{Corollary}
\newtheorem{theorem}[proposition]{Theorem}
%%%%%%%%%%%%%%%%%%%%%%%%%%%%%%%%%%%%%%%%%%%%%%%%%%%%%%%%%%%%
\theoremstyle{definition}
\newtheorem{definition}[proposition]{Definition}
\newtheorem{remark}[proposition]{Remark}
%%%%%%%%%%%%%%%%%%%%%%%%%%%%%%%%%%%%%%%%%%%%%%%%%%%%%%%%%%%%

\begin{document}
\title{Lectures on Random Matrices
(Spring 2025)
\\Lecture 11: Some universal asymptotics of Dyson Brownian Motion}


\date{March 26, 2025\footnote{\href{https://lpetrov.cc/rmt25/}{\texttt{Course webpage}}
$\bullet$ \href{https://lpetrov.cc/simulations/model/random-matrices/}{\texttt{Live simulations}}
$\bullet$ \href{https://lpetrov.cc/rmt25/rmt25-notes/rmt2025-l11.tex}{\texttt{TeX Source}}
$\bullet$
Updated at \currenttime, \today}}



\author{Leonid Petrov}


\maketitle

\section{Recap}

\subsection{Dyson Brownian Motion (DBM)}
We introduced a time-dependent model of random matrices by letting an
\(N\times N\) Hermitian matrix \(\mathcal{M}(t)\) evolve in time so that each
off-diagonal entry follows independent Brownian increments (real or complex
depending on the symmetry class).  Setting
\[
\mathcal{M}(t) \;=\; \frac{1}{\sqrt{2}}\bigl(X(t) + X^\dagger(t)\bigr),
\]
where \(X(t)\) is an \(N\times N\) matrix of i.i.d.\ Brownian motions,
produces a self-adjoint matrix with a stochastically evolving spectrum.
This model is full-rank matrix Brownian motion,
and works well for $\beta=1,2,4$.
For other $\beta$, we need an SDE to describe the evolution of the eigenvalues (particles).

\subsection{Eigenvalue SDE}
Denote by \(\lambda_1(t)\ge\cdots\ge\lambda_N(t)\) the ordered eigenvalues of
\(\mathcal{M}(t)\).  Dyson showed that these eigenvalues form a
continuous-time Markov process satisfying the SDE
\[
d\lambda_i(t)
\;=\;
\frac{\beta}{2}\,\sum_{j\neq i}\,\frac{dt}{\lambda_i(t)-\lambda_j(t)}
\;+\;
dW_i(t),
\quad
i=1,\dots,N,
\]
where \(\beta>0\) and \(W_i(t)\) are independent standard real Brownian motions.
For classical random matrix ensembles (\(\beta=1,2,4\)), this SDE describes how
the eigenvalues evolve under real symmetric (GOE), Hermitian (GUE), or
quaternionic (GSE) Brownian motion --- in the last \href{https://lpetrov.cc/rmt25/rmt25-notes/rmt2025-l10.pdf}{Lecture 10} we discussed the cases \(\beta=1,2\) in detail.
A key feature is the \emph{repulsion} term
\(\frac{1}{\lambda_i-\lambda_j}\), which prevents collisions (and ensures the
ordering remains intact).

\subsection{Preservation of G\(\boldsymbol{\beta}\)E density}
A fundamental result is that starting from all eigenvalues at \(0\),
the distribution of \(\lambda(t)\) at time \(t\) has the joint density
proportional to
\[
\prod_{i<j}|\lambda_i - \lambda_j|^\beta\;
\exp\Bigl\{-\tfrac{1}{2t}\sum_i \lambda_i^2\Bigr\},
\]
matching the Gaussian \(\beta\)-Ensemble (G\(\beta\)E) law.  Hence DBM provides
a dynamical realization of G\(\beta\)E.  Invariance can be checked by verifying
that this density is annihilated by the generator of the SDE.

\subsection{Harish--Chandra--Itzykson--Zuber (HCIZ) integral}
The HCIZ integral is a key tool for computing matrix integrals involving traces.
For two Hermitian matrices \(A\) and \(B\) with eigenvalues
\((a_1,\dots,a_N)\) and \((b_1,\dots,b_N)\), it states (in one common
normalization):
\[
\int_{U(N)} \exp\bigl(\mathrm{Tr}(A\,U\,B\,U^\dagger)\bigr)\,dU
\;=\;
\prod_{k=1}^{N-1} k!\;
\frac{\det\bigl[e^{\,a_i b_j}\bigr]_{i,j=1}^N}{
\prod_{1\le i<j\le N}(a_j-a_i)\,\prod_{1\le i<j\le N}(b_j-b_i)}\,.
\]
This formula is instrumental in deriving transition densities for
\(\beta=2\) Dyson Brownian Motion.  

\section{Optional: proof of HCIZ integral via representation theory}



\section{Determinantal structure for $\beta=2$}

\subsection{Transition density}

\begin{theorem}[\(\beta=2\) Dyson Brownian Motion Transition Probabilities]
	\label{thm:dbm-transition}
For \(\beta=2\), let \(\lambda(t)=(\lambda_1(t)\ge \cdots \ge \lambda_N(t))\) follow Dyson Brownian Motion starting at \(\lambda(0)=\mathbf{a}=(a_1\ge \cdots \ge a_N)\).  Then for each fixed time \(t>0\),
\[
P\bigl(\lambda(t) = \mathbf{x}\;\big|\;\lambda(0)=\mathbf{a}\bigr)
\;=\;
N!\,\bigl(\frac{1}{\sqrt{2\pi t}}\bigr)^{N}
\;\prod_{1\le i<j\le N}\frac{x_i - x_j}{a_i - a_j}
\;\det\Bigl[\exp\Bigl(-\frac{(x_i - a_j)^2}{2t}\Bigr)\Bigr]_{i,j=1}^N,
\]
where \(x_1 \ge \dots \ge x_N\).
\end{theorem}

\begin{proof}
Consider an \(N\times N\) Hermitian matrix process \(X(t)\) whose entries perform independent complex Brownian motions (so that \(X(t)\) is distributed as \(A + \sqrt{t}\,\mathrm{GUE}\) at each fixed time, with \(A=\mathrm{diag}(a_1,\dots,a_N)\)).  Its eigenvalues \(\lambda_1(t)\ge \cdots \ge \lambda_N(t)\) evolve exactly according to the \(\beta=2\) Dyson Brownian Motion.

The density of \(X\) at time \(t\), viewed as a random matrix, is proportional to
\[
\exp\Bigl(-\tfrac{1}{2t}\,\mathrm{Tr}\bigl(X-A\bigr)^2\Bigr).
\]
If we replace \(A\) by \(U\,A\,U^\dagger\) for any fixed unitary \(U\), the law of \(X\) remains the same (this follows from the unitary invariance of the GUE).  Thus the distribution of the eigenvalues of \(X\) is unchanged by such conjugation.

One writes
\[
\int_{U(N)}
\exp\Bigl(-\tfrac{1}{2t}\,\mathrm{Tr}\bigl(X-U\,A\,U^\dagger\bigr)^2\Bigr)\,dU
\;=\;
\text{(const.)} \times
\text{[HCIZ integral in the variables }(X,A)\text{]},
\]
which by the Harish--Chandra--Itzykson--Zuber formula leads to a product of determinants and a factor that is precisely
\[
\exp\Bigl(-\tfrac{1}{2t}\sum_{i=1}^N x_i^2
- \tfrac{1}{2t}\sum_{i=1}^N a_i^2\Bigr)\,
\frac{\det\Bigl[\exp\bigl(\tfrac{x_i\,a_j}{t}\bigr)\Bigr]}{\prod_{i<j}(x_i - x_j)\,(a_i - a_j)},
\]
where \(x_1,\dots,x_N\) are the eigenvalues of \(X\).

To convert this matrix distribution into a distribution on eigenvalues alone, we multiply by the usual Vandermonde Jacobian
\(\prod_{i<j}(x_i - x_j)^2\)
(which comes from integrating out the unitary degrees of freedom).  This produces exactly
\[
N!\,\bigl(\tfrac{1}{\sqrt{2\pi t}}\bigr)^{N}\,
\prod_{i<j}\frac{x_i - x_j}{a_i - a_j}
\,\det\Bigl[\exp\Bigl(-\tfrac{(x_i - a_j)^2}{2t}\Bigr)\Bigr].
\]
Hence we obtain the stated transition probability for the Dyson Brownian Motion at \(\beta=2\).
\end{proof}

\begin{remark}
The factor \(N!\,(\tfrac{1}{\sqrt{2\pi t}})^N\) arises naturally from normalizing the Gaussian increments and accounts for the ordering \(\lambda_1\ge\cdots\ge \lambda_N\).  The determinant and product factors encode the eigenvalue ``repulsion'' characteristic of \(\beta=2\) random matrices.
\end{remark}


\subsection{Determinantal correlations}

\begin{theorem}[Determinantal structure for $\beta=2$ DBM]
\label{thm:dbm-det-kernel}
Let $\{x_1(t),\dots,x_n(t)\}$ be the eigenvalues at time $t>0$ of the $\beta=2$ Dyson Brownian Motion started at initial locations $(a_1,\dots,a_n)$ at time $0$.  Equivalently, these $x_i(t)$ are the eigenvalues of
\[
A + \sqrt{t}\,G,
\]
where $A=\mathrm{diag}(a_1,\dots,a_n)$ and $G$ is a random Hermitian matrix from the GUE.  Then the (random) point configuration $\{x_i(t)\}$ forms a determinantal point process with correlation kernel
\[
K_t(x,y)
=
\frac{1}{(2\pi i)^2\,t}
\oint \oint
\exp\Bigl(\frac{w^2 - 2\,y\,w}{2\,t}\Bigr)\,
\biggl/
\exp\Bigl(\frac{z^2 - 2\,x\,z}{2\,t}\Bigr)
\;\prod_{i=1}^n \frac{w - a_i}{z - a_i}
\;\frac{dw\,dz}{w - z}.
\]
Here $z$ goes around all the points $a_1,\ldots,a_n $,
and the $w$ contour
passes from $-i\infty$ to $i\infty$, to the right of the $z$ contour.
\end{theorem}

\begin{itemize}
	\item If $a_1=\dots=a_n=0$ and $t=1$, this kernel reduces to the familiar correlation kernel of the GUE (see \href{https://lpetrov.cc/rmt25/rmt25-notes/rmt2025-l06.pdf}{Lecture 6}).
\item
	One can use this formula to study the Baik--Ben
	Arous--P\'ech\'e (BBP)
	\cite{BBP2005phase}
	phase transition for $\beta=2$,
	which deals with finite rank perturbations of the GUE random matrix ensemble.
	Indeed, rank $r$ perturbation corresponds to taking $a_1,\ldots,a_r\ne0 $,
	and $a_{r+1}=\dots=a_n=0$.
\end{itemize}

\subsection{On the proof of determinantal structure}

The idea of the proof of \Cref{thm:dbm-det-kernel} is to
represent the measure (the transition density) as a
product of determinants. In general,
if a measure is given as a product of determinants,
there is a well-studied method (biorthogonal ensembles and,
more generally, the Eynard--Mehta theorem)
to compute the determinantal correlation kernel.
We refer to
\cite{borodin2005eynard},
\cite{Borodin2009} for a detailed exposition
in the discrete case (which is arguably more transparent).
The first step for the Dyson Brownian Motion is as follows.

\begin{lemma}[Density representation]
\label{lem:density_representation}
Let $P_t(x\to y)$ be the transition probability kernel of standard Brownian motion,
\[
   P_t(x\to y) \;=\; \frac{1}{\sqrt{2\pi\,t}}\,
   \exp\Bigl(-\tfrac{(x-y)^2}{2\,t}\Bigr).
\]
Then the density of the eigenvalues $(x_1,\dots,x_N)$
of DBM started at $(a_1,\dots,a_N)$ at time $0$
admits the representation
\begin{equation}
	\label{eq:density_representation}
   \lim_{s\to\infty}
   \biggl(\frac1Z\biggr)\,
   \det\Bigl[P_t\bigl(a_i\to x_j\bigr)\Bigr]_{i,j=1}^N
   \,\det\Bigl[P_s\bigl(x_i\to k-1\bigr)\Bigr]_{i,k=1}^N.
 \end{equation}
\end{lemma}
\begin{remark}
	This representation
	\eqref{eq:density_representation}
	is related to an alternative description of the
	$\beta=2$ Dyson Brownian Motion as
	an ensemble of noncolliding Brownian motions
	(that is, independent Brownian motions, conditioned to never collide).
\end{remark}

\begin{proof}[Proof of \Cref{lem:density_representation}]
The first determinant (as $s\to\infty$) matches the determinant
we have in \Cref{thm:dbm-transition}.
It remains to analyze the second determinant
\[
   \det\Bigl[
      P_s\bigl(x_j \to k-1\bigr)
   \Bigr]_{j,k=1}^N
   \;=\;
   \det\Bigl[
      \tfrac{1}{\sqrt{2\pi\,s}}\,
      \exp\Bigl(-\tfrac{\bigl((k-1) - x_j\bigr)^2}{2\,s}\Bigr)
   \Bigr]_{j,k=1}^N.
\]
We may ignore the factor \(\tfrac{1}{\sqrt{2\pi\,s}}\) in each entry since it does not depend on \(x_j\).  Inside the exponential,
\[
   -\,\frac{\bigl((k-1) - x_j\bigr)^2}{2\,s}
   \;=\;
   -\,\frac{x_j^2}{2\,s}
   \;+\;\frac{x_j\,(k-1)}{s}
   \;-\;\frac{(k-1)^2}{2\,s}.
\]
Thus, up to the factor
\(\exp\bigl(-\,\tfrac{(k-1)^2}{2\,s}\bigr)\)
(which depends only on \(k\) and hence is independent of each \(x_j\)),
we can factor out
\(\exp\bigl(-\,\tfrac{x_j^2}{2\,s}\bigr)\)
from row \(j\).  Consequently, the nontrivial part of the determinant becomes
\[
   \det\Bigl[
      e^{\,\frac{x_j\,(k-1)}{s}}
   \Bigr]_{j,k=1}^N.
\]
Recognize this as a Vandermonde-type determinant in the variables \(e^{\,x_j/s}\).  Indeed,
\[
   \det\Bigl[
      e^{\,\frac{x_j\,(k-1)}{s}}
   \Bigr]_{j,k=1}^N
   \;=\;
   \prod_{1 \le i<j \le N}
   \Bigl(e^{\,\frac{x_i}{s}} - e^{\,\frac{x_j}{s}}\Bigr).
\]
As \(s \to \infty\), we expand
\(e^{\,\frac{x_i}{s}} = 1 + \frac{x_i}{s} + O\bigl(\tfrac{1}{s^2}\bigr)\),
so each difference
\(\bigl(e^{\,\frac{x_i}{s}} - e^{\,\frac{x_j}{s}}\bigr)
 \sim \tfrac{x_i - x_j}{s}\).
Hence,
\[
   \prod_{1 \le i<j \le N}
   \Bigl(e^{\,\frac{x_i}{s}} - e^{\,\frac{x_j}{s}}\Bigr)
   \;\sim\;
   \frac{1}{s^{\,\frac{N(N-1)}{2}}}
   \prod_{1 \le i<j \le N} (x_i - x_j).
\]
Combining all these factors and matching with the first determinant (as $s\to\infty$) verifies the claimed product form, up to overall constants that do not depend on the variables \(x_j\).  This completes the proof.
\end{proof}

Then, the product of determinants idea
(biorthogonal ensembles)
applies
to the density \eqref{eq:density_representation}
before the limit $s\to\infty$,
and simplifies after taking the limit.
We omit the details here,
see~Problem~\ref{prob:biorthogonal}.






















\appendix
\setcounter{section}{10}

\section{Problems (due 2025-04-29)}


\subsection{Biorthogonal ensembles}
\label{prob:biorthogonal}

Derive \Cref{thm:dbm-det-kernel} from
\Cref{lem:density_representation}
using the orthogonalization process similar
to~\href{https://lpetrov.cc/rmt25/rmt25-notes/rmt2025-l05.pdf}{Lecture 5},
and then taking the limit as $s\to\infty$.

\subsection{Scaling of the kernel}
\label{prob:scaling}

Let $a_i=0$ in \Cref{thm:dbm-det-kernel}.
Find $\alpha$ such that
$t^\alpha K_t(x/\sqrt{t},y/\sqrt{t})$ is independent of $t$.
Can you explain this value of $\alpha$?





\bibliographystyle{alpha}
\bibliography{bib}


\medskip

\textsc{L. Petrov, University of Virginia, Department of Mathematics, 141 Cabell Drive, Kerchof Hall, P.O. Box 400137, Charlottesville, VA 22904, USA}

E-mail: \texttt{lenia.petrov@gmail.com}


\end{document}
