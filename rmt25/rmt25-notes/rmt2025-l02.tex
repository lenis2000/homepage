\documentclass[letterpaper,11pt,oneside,reqno]{article}

%%%%%%%%%%%%%%%%%%%%%%%%%%%%%%%%%%%%%%%%%%%%%%%%%%%%%%%%%%%%

\usepackage[pdftex,backref=page,colorlinks=true,linkcolor=blue,citecolor=red]{hyperref}
\usepackage[alphabetic,nobysame]{amsrefs}

%%%%%%%%%%%%%%%%%%%%%%%%%%%%%%%%%%%%%%%%%%%%%%%%%%%%%%%%%%%%
%main packages
\usepackage{amsmath,amssymb,amsthm,amsfonts,mathtools}
\usepackage{graphicx,color}
\usepackage{upgreek}
\usepackage[mathscr]{euscript}

%equations
\allowdisplaybreaks
\numberwithin{equation}{section}

%tikz
\usepackage{tikz}
\usetikzlibrary{shapes,arrows,positioning,decorations.markings}

%conveniences
\usepackage{array}
\usepackage{adjustbox}
\usepackage{cleveref}
\usepackage{enumerate}
\usepackage{datetime}
\usepackage{comment}

%paper geometry
\usepackage[DIV=12]{typearea}

%%%%%%%%%%%%%%%%%%%%%%%%%%%%%%%%%%%%%%%%%%%%%%%%%%%%%%%%%%%%
%draft-specific
\synctex=1
% \usepackage{refcheck,comment}

%%%%%%%%%%%%%%%%%%%%%%%%%%%%%%%%%%%%%%%%%%%%%%%%%%%%%%%%%%%%
%this paper specific
\newcommand{\ssp}{\hspace{1pt}}

%%%%%%%%%%%%%%%%%%%%%%%%%%%%%%%%%%%%%%%%%%%%%%%%%%%%%%%%%%%%
\newtheorem{proposition}{Proposition}[section]
\newtheorem{lemma}[proposition]{Lemma}
\newtheorem{corollary}[proposition]{Corollary}
\newtheorem{theorem}[proposition]{Theorem}
%%%%%%%%%%%%%%%%%%%%%%%%%%%%%%%%%%%%%%%%%%%%%%%%%%%%%%%%%%%%
\theoremstyle{definition}
\newtheorem{definition}[proposition]{Definition}
\newtheorem{remark}[proposition]{Remark}
%%%%%%%%%%%%%%%%%%%%%%%%%%%%%%%%%%%%%%%%%%%%%%%%%%%%%%%%%%%%

\newenvironment{lnotes}{\section*{Notes for the lecturer}}{}


\begin{document}
\title{Lectures on Random Matrices
(Spring 2025) \\Lecture 2: Wigner semicircle law}


\date{Wednesday, January 15, 2025\footnote{\href{https://lpetrov.cc/rmt25/}{\texttt{Course webpage}}
$\bullet$ \href{https://lpetrov.cc/rmt25/rmt25-notes/rmt2025-l01.tex}{\texttt{TeX Source}}
$\bullet$
Updated at \currenttime, \today}}



\author{Leonid Petrov}


\maketitle



\begin{lnotes}
	PREP:
	\begin{enumerate}
		\item Start:
			Catalan number formula
		\item Moments of SC need to be computed
		\item 
			SC is uniquely determined by its moments; need Carleman criterion to show that the moments determine the distribution.
		\item from expected moments to the full convergence, some analysis needed
	\end{enumerate}
\end{lnotes}


\section{Recap}

We are working on the Wigner semicircle law. 
\begin{enumerate}
	\item Wigner matrices $W$:
		real symmetric random matrices with iid entries 
		$X_{ij}$, $i>j$ (mean 0, variance $\sigma^2$);
		and iid diagonal entries $X_{ii}$ (mean 0, some other variance and distribution).
	\item Empirical spectral distribution (ESD)
		\begin{equation*}
			\nu_n = \frac{1}{n} \sum_{i=1}^{n} \delta_{\lambda_i/\sqrt{n}},
		\end{equation*}
		which is a random probability measure on $\mathbb{R}$.
	\item Semicircle distribution $\mu_{\mathrm{sc}}$:
		\begin{equation*}
			\mu_{\mathrm{sc}}(dx) = \frac{1}{2\pi} \sqrt{4-x^2} \, dx,
			\qquad x \in [-2,2].
		\end{equation*}
	\item Computation of expected traces of powers of $W$. We
		showed that
		\begin{equation*}
			\int_{\mathbb{R}}x^k \nu_n(dx)\to
			\#\left\{ \textnormal{rooted planar trees with $k/2$ edges} \right\}.
		\end{equation*}
\end{enumerate}

\section{Two computations}

First, we finish the combinatorial part,
and match the limiting expected traces of powers of $W$
to moments of the semicircle law.

\subsection{Moments of the semicircle law}

We also need to match the Catalan numbers to the moments of the semicircle law.
Let $k=2m$, and we need to compute
the integral
\begin{equation*}
	\int_{-2}^{2} x^{2m} \frac{1}{2\pi} \sqrt{4-x^2} \, dx.
\end{equation*}

By symmetry, we write:
\[
\int_{-2}^2 x^{2m}\rho(x)\, dx = \frac{2}{\pi} \int_0^2 x^{2m} \sqrt{4-x^2}\, dx.
\]

Using the substitution \(x = 2\sin\theta\), we have \(dx = 2\cos\theta\, d\theta\). The integral becomes:
\[
\frac{2}{\pi} \int_0^{\pi/2} (2\sin\theta)^{2m} (2\cos\theta) (2\cos\theta\, d\theta)
= \frac{2^{2m+2}}{\pi} \int_0^{\pi/2} \sin^{2m}\theta \cos^2\theta\, d\theta.
\]
Using \(\cos^2\theta = 1 - \sin^2\theta\), we split the integral:
\[
\frac{2^{2m+2}}{\pi} \left( \int_0^{\pi/2} \sin^{2m}\theta\, d\theta - \int_0^{\pi/2} \sin^{2m+2}\theta\, d\theta \right).
\]
Using the standard formula
\begin{equation}
	\label{eq:sin-integral}
\int_0^{\pi/2} \sin^{2n}\theta\, d\theta = \frac{\pi}{2} \frac{(2n)!}{2^{2n} (n!)^2},
\end{equation}
we compute each term:
\[
\frac{2^{2m+2}}{\pi} \left( \frac{\pi}{2} \frac{(2m)!}{2^{2m}(m!)^2} - \frac{\pi}{2} \frac{(2m+2)!}{2^{2m+2}((m+1)!)^2} \right).
\]
After simplification, this becomes
$C_m$, the $m$-th Catalan number.

\subsection{Counting trees and Catalan numbers}




\section{Analysis steps in the proof}

We are done with combinatorics, and it remains to justify that
the computations lead to the desired semicircle law
from \href{https://lpetrov.cc/rmt25/rmt25-notes/rmt2025-l01.pdf}{Lecture 1}.




\appendix
\setcounter{section}{0}

\section{Problems (due 2025-02-15)}

\subsection{Standard formula}

Prove formula \eqref{eq:sin-integral}:
\begin{equation*}
	\int_0^{\pi/2} \sin^{2n}\theta\, d\theta = \frac{\pi}{2} \frac{(2n)!}{2^{2n} (n!)^2}.
\end{equation*}




\bibliographystyle{alpha}
\bibliography{bib}


\medskip

\textsc{L. Petrov, University of Virginia, Department of Mathematics, 141 Cabell Drive, Kerchof Hall, P.O. Box 400137, Charlottesville, VA 22904, USA}

E-mail: \texttt{lenia.petrov@gmail.com}


\end{document}
