\documentclass[letterpaper,11pt,oneside,reqno]{article}

%%%%%%%%%%%%%%%%%%%%%%%%%%%%%%%%%%%%%%%%%%%%%%%%%%%%%%%%%%%%

\usepackage[pdftex,backref=page,colorlinks=true,linkcolor=blue,citecolor=red]{hyperref}
\usepackage[alphabetic,nobysame]{amsrefs}

%%%%%%%%%%%%%%%%%%%%%%%%%%%%%%%%%%%%%%%%%%%%%%%%%%%%%%%%%%%%
%main packages
\usepackage{amsmath,amssymb,amsthm,amsfonts,mathtools}
\usepackage{graphicx,color}
\usepackage{upgreek}
\usepackage[mathscr]{euscript}

%equations
\allowdisplaybreaks
\numberwithin{equation}{section}

%tikz
\usepackage{tikz}
\usetikzlibrary{shapes,arrows,positioning,decorations.markings}

%conveniences
\usepackage{array}
\usepackage{adjustbox}
\usepackage{cleveref}
\usepackage{enumerate}
\usepackage{datetime}

%paper geometry
\usepackage[DIV=12]{typearea}

%%%%%%%%%%%%%%%%%%%%%%%%%%%%%%%%%%%%%%%%%%%%%%%%%%%%%%%%%%%%
%draft-specific
\synctex=1
% \usepackage{refcheck,comment}

%%%%%%%%%%%%%%%%%%%%%%%%%%%%%%%%%%%%%%%%%%%%%%%%%%%%%%%%%%%%
%this paper specific
\newcommand{\ssp}{\hspace{1pt}}

%%%%%%%%%%%%%%%%%%%%%%%%%%%%%%%%%%%%%%%%%%%%%%%%%%%%%%%%%%%%
\newtheorem{proposition}{Proposition}[section]
\newtheorem{lemma}[proposition]{Lemma}
\newtheorem{corollary}[proposition]{Corollary}
\newtheorem{theorem}[proposition]{Theorem}
%%%%%%%%%%%%%%%%%%%%%%%%%%%%%%%%%%%%%%%%%%%%%%%%%%%%%%%%%%%%
\theoremstyle{definition}
\newtheorem{definition}[proposition]{Definition}
\newtheorem{remark}[proposition]{Remark}
%%%%%%%%%%%%%%%%%%%%%%%%%%%%%%%%%%%%%%%%%%%%%%%%%%%%%%%%%%%%

\begin{document}
\title{Lectures on Random Matrices
(Spring 2025)
\\Lecture 14: Title TBD}


\date{Wednesday, April 16, 2025\footnote{\href{https://lpetrov.cc/rmt25/}{\texttt{Course webpage}}
$\bullet$ \href{https://lpetrov.cc/simulations/model/random-matrices/}{\texttt{Live simulations}}
$\bullet$ \href{https://lpetrov.cc/rmt25/rmt25-notes/rmt2025-l14.tex}{\texttt{TeX Source}}
$\bullet$
Updated at \currenttime, \today}}



\author{Leonid Petrov}


\maketitle



recap:

1. what are we proving

2. what is the RSK toggle doing

3. weight preservation (new)

4. now, apply this to the geometric LPP weights

































\appendix
\setcounter{section}{13}

\section{Problems (due 2025-04-29)}





\bibliographystyle{alpha}
\bibliography{bib}


\medskip

\textsc{L. Petrov, University of Virginia, Department of Mathematics, 141 Cabell Drive, Kerchof Hall, P.O. Box 400137, Charlottesville, VA 22904, USA}

E-mail: \texttt{lenia.petrov@gmail.com}


\end{document}
