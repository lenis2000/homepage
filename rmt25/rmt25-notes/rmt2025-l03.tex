\documentclass[letterpaper,11pt,oneside,reqno]{article}

%%%%%%%%%%%%%%%%%%%%%%%%%%%%%%%%%%%%%%%%%%%%%%%%%%%%%%%%%%%%

\usepackage[pdftex,backref=page,colorlinks=true,linkcolor=blue,citecolor=red]{hyperref}
\usepackage[alphabetic,nobysame]{amsrefs}

%%%%%%%%%%%%%%%%%%%%%%%%%%%%%%%%%%%%%%%%%%%%%%%%%%%%%%%%%%%%
%main packages
\usepackage{amsmath,amssymb,amsthm,amsfonts,mathtools}
\usepackage{graphicx,color}
\usepackage{upgreek}
\usepackage[mathscr]{euscript}

%equations
\allowdisplaybreaks
\numberwithin{equation}{section}

%tikz
\usepackage{tikz}
\usetikzlibrary{shapes,arrows,positioning,decorations.markings}

%conveniences
\usepackage{array}
\usepackage{adjustbox}
\usepackage{cleveref}
\usepackage{enumerate}
\usepackage{datetime}
\usepackage{comment}

%paper geometry
\usepackage[DIV=12]{typearea}

%%%%%%%%%%%%%%%%%%%%%%%%%%%%%%%%%%%%%%%%%%%%%%%%%%%%%%%%%%%%
%draft-specific
\synctex=1
% \usepackage{refcheck,comment}

%%%%%%%%%%%%%%%%%%%%%%%%%%%%%%%%%%%%%%%%%%%%%%%%%%%%%%%%%%%%
%this paper specific
\newcommand{\ssp}{\hspace{1pt}}

%%%%%%%%%%%%%%%%%%%%%%%%%%%%%%%%%%%%%%%%%%%%%%%%%%%%%%%%%%%%
\newtheorem{proposition}{Proposition}[section]
\newtheorem{lemma}[proposition]{Lemma}
\newtheorem{corollary}[proposition]{Corollary}
\newtheorem{theorem}[proposition]{Theorem}
%%%%%%%%%%%%%%%%%%%%%%%%%%%%%%%%%%%%%%%%%%%%%%%%%%%%%%%%%%%%
\theoremstyle{definition}
\newtheorem{definition}[proposition]{Definition}
\newtheorem{remark}[proposition]{Remark}
%%%%%%%%%%%%%%%%%%%%%%%%%%%%%%%%%%%%%%%%%%%%%%%%%%%%%%%%%%%%

\newenvironment{lnotes}{\section*{Notes for the lecturer}}{}
% \excludecomment{lnotes}


\begin{document}
\title{Lectures on Random Matrices
(Spring 2025)
\\Lecture 3: Gaussian and tridiagonal matrices}


\date{Wednesday, January 22, 2025\footnote{\href{https://lpetrov.cc/rmt25/}{\texttt{Course webpage}}
$\bullet$ \href{https://lpetrov.cc/rmt25/rmt25-notes/rmt2025-l03.tex}{\texttt{TeX Source}}
$\bullet$
Updated at \currenttime, \today}}

\author{Leonid Petrov}


\maketitle


\tableofcontents




\section{Recap}

We have established the semicircle law for
real Wigner random matrices.
If $W$ is an $n\times n$ real symmetric matrix with
independent entries $X_{ij}$ above the main diagonal
(mean zero, variance~$1$), and mean zero diagonal entries,
then the empirical spectral distribution of $W/\sqrt{n}$
converges to the semicircle law as $n\to\infty$:
\begin{equation}
	\label{eq:semicircle_conv}
	\lim_{n\to\infty} \frac{1}{n} \sum_{i=1}^n \delta_{\lambda_i/\sqrt n} =
	\mu_{\mathrm{sc}},
\end{equation}
where
\begin{equation*}
	\mu_{\mathrm{sc}}(dx) = \begin{cases}
		\frac{1}{2\pi} \sqrt{4-x^2} \, dx, & \text{if } |x|\le 2, \\
		0, & \text{otherwise}.
	\end{cases}
\end{equation*}
The convergence
in \eqref{eq:semicircle_conv} is weakly almost sure.
The way we got the result is by expanding
$\operatorname{\mathbb{E}}\operatorname{Tr} (W^k)$ and counting
trees, plus analytic lemmas which ensure that
the convergence of expected powers of traces is enough
to conclude the convergence
\eqref{eq:semicircle_conv}
of the empirical spectral measures.

\medskip

Today, we are going to focus on Gaussian ensembles. The plan is:
\begin{enumerate}[$\bullet$]
	\item Definition and spectral density for real symmetric Gaussian matrices (GOE).
	\item Tridiagonalization and general beta ensemble.
	\item Wigner's semicircle law via tridiagonalization.
\end{enumerate}

\section{Gaussian ensembles}

\subsection{Definitions}
\label{sub:GOE_GUE_definitons}

Recall that a real Wigner matrix $W$ can be modeled as
\begin{equation*}
	W=\frac{Y+Y^\top}{\sqrt{2}},
\end{equation*}
where $Y$ is an $n\times n$ matrix with independent entries $Y_{ij}$,
$1\le i,j\le n$, such that $Y_{ij}$ are mean zero, variance~$1$.
Then for $1\le i<j\le n$, we have for the matrix
$W=(X_{ij})$:
\begin{equation*}
	\operatorname{\mathrm{Var}}\left( X_{ii} \right)=
	\operatorname{\mathrm{Var}}( \sqrt 2\ssp Y_{ii} )=2,\qquad
	\operatorname{\mathrm{Var}}\left( X_{ij} \right)=
	\operatorname{\mathrm{Var}}\left( \frac{Y_{ij}+Y_{ji}}{\sqrt 2} \right)=1.
\end{equation*}

If, in addition, we assume that $Y_{ij}$ are standard Gaussian
$\mathcal{N}(0,1)$, then the distribution of $W$ is called
the \emph{Gaussian Orthogonal Ensemble} (GOE).

For the complex case, we
have the \emph{standard complex Gaussian random variable}
\begin{equation*}
	Z=\frac{1}{\sqrt 2}\left( Z^R+\mathbf{i}\ssp Z^I \right),
	\qquad
	\operatorname{\mathbb{E}} (Z)=0,
	\qquad
	\operatorname{\mathrm{Var}}_{\mathbb{C}}(Z)\coloneqq
	\operatorname{\mathbb{E}} (|Z|^2)=
	\frac{
	\operatorname{\mathbb{E}} (|Z^R|^2)+
	\operatorname{\mathbb{E}} (|Z^I|^2)}{2}=1
	,
\end{equation*}
where $Z^R$ and $Z^I$ are independent
standard Gaussian real random variables $\mathcal{N}(0,1)$.

If we take $Y$ to be an $n\times n$ matrix with independent
entries $Y_{ij}$, $1\le i,j\le n$
distributed as $Z$, then the random matrix\footnote{$Y^\dagger$ denotes the transpose of $Y$ combined with complex conjugation.}
\begin{equation*}
	W=\frac{Y+Y^\dagger}{\sqrt 2}
\end{equation*}
is said to have the \emph{Gaussian Unitary Ensemble} (GUE) distribution.
For the GUE matrix $W=(X_{ij})$,
we have for $1\le i<j\le n$:
\begin{equation*}
	\operatorname{\mathrm{Var}}_{\mathbb{C}}(X_{ii})=2,
	\qquad
	\operatorname{\mathrm{Var}}_{\mathbb{C}}(X_{ij})
	=\frac{1}{4}
	\Bigl[ \operatorname{\mathbb{E}}(Z_{ij}^R
			+
		Z_{ji}^R)^2
		+
		\operatorname{\mathbb{E}}(Z_{ij}^I
			+
		Z_{ji}^I)^2
	\Bigr]
	=1.
\end{equation*}

Both GOE and GUE have real eigenvalues $\lambda_1 \ge \ldots \ge \lambda_n$.
We are going to describe the joint distribution of these eigenvalues.
Despite the fact that the map from a matrix to its eigenvalues
is quite complicated and nonlinear (you need to solve an equation of degree $n$),
the distribution of eigenvalues in the Gaussian cases is fully explicit.

See Problem \ref{prob:invariance_GOE_GUE}
for invariance of GOE/GUE under orthogonal/unitary conjugation
(this is where the names ``orthogonal'' and ``unitary'' come from).

\begin{remark}
	There is a third player in the game, the \emph{Gaussian
	Symplectic Ensemble} (GSE),which we will mainly ignore in
	this course
	due to its less intuitive quaternionic nature.
\end{remark}

%%%%%%%%%%%%%%%%%%%%%%%%%%%%%%%%%%%%%%%%%%%%%%%%%%%%%%%%%%%%
\subsection{Joint eigenvalue distribution for GOE}
\label{sub:GOE-derivation}
%%%%%%%%%%%%%%%%%%%%%%%%%%%%%%%%%%%%%%%%%%%%%%%%%%%%%%%%%%%%

In this section, we give a derivation of the joint probability density for the GOE.

\begin{theorem}[GOE Joint Eigenvalue Density]
\label{thm:GOE-joint-eigs-detailed}
Let \(W\) be an \(n\times n\) real symmetric matrix with
the GOE distribution (\Cref{sub:GOE_GUE_definitons}).
Then its ordered real eigenvalues \(\lambda_1 \le \cdots \le
\lambda_n\)
of $W/\sqrt 2$
have a joint probability density function
on $\mathbb{R}^n$
given by:
\[
  p(\lambda_1,\ldots,\lambda_n)
  \;=\;\frac{1}{Z_{n}}
  \,\prod_{1 \le i < j \le n}
  \!\!\bigl|\lambda_i - \lambda_j\bigr|\,
  \exp\Bigl(
    -\frac{1}{2}\sum_{k=1}^n \lambda_k^2
  \Bigr),
\]
where \(Z_{n}\) is a constant (depending on \(n\) but not on \(\lambda_i\)) ensuring the density integrates to 1:
\begin{equation*}
	Z_n=Z_n^{GOE}=\frac{(2\pi)^{n/2}}{n!}
	\prod_{j=0}^{n-1}\frac{\Gamma(1+(j+1)\beta/2)}{\Gamma(1+\beta/2)}, \qquad
	\beta=1.
\end{equation*}
\end{theorem}
\begin{remark}
	We renormalized the GOE by a factor of $\sqrt 2$ to make the
	Gaussian part of the density, $\exp(-\frac{1}{2}\sum_{k=1}^n \lambda_k^2)$,
	standard. In the GUE case, no normalization is required.
\end{remark}

We break the proof into four major steps,
considered in
\Cref{subsec:density-entries,subsec:spectral,subsec:jacobian,subsec:final-form}
below.

\subsection{Step A. Joint density of matrix entries}
\label{subsec:density-entries}

Let us label all independent entries of \(W/\sqrt 2\):
\[
	\{\underbrace{X_{12}, X_{13},\dots, X_{23},\ldots }_{\text{above diag}},
	\underbrace{X_{22}, X_{33},\dots}_{\text{diag}}\}.
\]
There are \(\frac{n(n-1)}{2}\) off-diagonal entries
with variance $1/2$,
and \(n\) diagonal entries with variance $1$.
The joint density of these entries (ignoring normalization for a moment) is
proportional to
\begin{equation}
	\label{eq:GOE-density-entries-first}
	f(x_{12},x_{13},\ldots,x_{22},x_{33},\ldots )
  \propto\exp\Bigl(
    - \sum_{i<j} x_{ij}^2
    -\frac{1}{2} \sum_{i=1}^n x_{ii}^2
  \Bigr)=
	\exp
	\Bigl( -\frac{1}{2}\sum_{i,j=1}^n x_{ij}^2 \Bigr),
\end{equation}
where in the right-hand side, we have
$x_{ij}=x_{ji}$ for $i\ne j$.
We then recognize
\[
	\sum_{i,j=1}^n x_{ij}^2=\operatorname{Tr}(W^2)=\sum_{k=1}^n \lambda_k^2.
\]
Including the normalization for Gaussians, one arrives at
the density on $\mathbb{R}^{n(n+1)/2}$:
\[
  f(W)\,dW
  \;=\;
  \pi^{-\tfrac{n(n-1)}{4}}
  \,\bigl(2\pi\bigr)^{-\tfrac{n}{4}}
  \,\exp\Bigl(-\tfrac{1}{2}\operatorname{Tr}(W^2)\Bigr)\; dW,
\]
where \(dW\) is the product measure over the \(\tfrac{n(n+1)}{2}\) independent entries.

\subsection{Step B. Spectral decomposition}
\label{subsec:spectral}

Since \(W\) is real symmetric, it can be orthogonally diagonalized:
\[
  W = Q\,\Lambda\,Q^\top,\quad
  Q \in O(n),
\]
where \(\Lambda = \mathrm{diag}(\lambda_1,\ldots,\lambda_n)\) has the eigenvalues.  Then, as we saw before, we have
\[
  \operatorname{Tr}(W^2)
  = \operatorname{Tr}\bigl(Q\,\Lambda\,Q^\top Q\,\Lambda\,Q^\top\bigr)
  = \operatorname{Tr}(\Lambda^2)
  = \sum_{k=1}^n \lambda_k^2.
\]
The map
from $W$ to $(\Lambda,Q)$ is not one-to one,
but in case $W$ has distinct eigenvalues,
the preimage of $(\Lambda,Q)$
contains $2^n$ elements.
See Problems~\ref{prob:GOE-preimage} and \ref{prob:distinct-eigenvalues}.
\medskip

It remains to make the change of variables from \(W\) to \(\Lambda\), which involves the Jacobian.

\subsection{Step C. Jacobian}
\label{subsec:jacobian}

We now examine how the measure \(dW\) in the space of real symmetric matrices factors into a piece depending on \(\{\lambda_i\}\) and a piece depending on \(Q\).  Formally,
\[
  dW
  = \Bigl|\det\bigl(\tfrac{\partial W}{\partial(\Lambda,Q)}\bigr)\Bigr|
    \,d\Lambda\,dQ,
\]
where \(dQ\) is the Haar measure\footnote{Recall that the
	Haar measure on \(O(n)\) is the unique
	(up to a constant factor) measure that is invariant under
	group shifts (in this situation, both left and right shifts
	work). In probabilistic terms,
	if a random orthogonal matrix $Q$ is Haar-distributed,
	then $QR$ and $RQ$ are also Haar-distributed for any fixed orthogonal
matrix $R$.}
on \(O(n)\), and
\(d\Lambda\) is the Lebesgue measure on \(\mathbb{R}^n\).
The Lebesgue measure later needs to be restricted
to the ``Weyl chamber''
\(\lambda_1\le \cdots\le \lambda_n\) if we want an ordering,
this introduces the simple factor \(n!\) in the final density.


\begin{lemma}[Jacobian for Spectral Decomposition]
\label{lemma:Jacobian-GOE}
For real symmetric \(W=Q\Lambda Q^\top\), one has
\[
  \bigl|\det\bigl(\tfrac{\partial W}{\partial(\Lambda,Q)}\bigr)\bigr|
  \;=\;
	\mathrm{const}
  \prod_{1\le i<j\le n}
	\!\!
	\bigl|\lambda_i - \lambda_j\bigr|,
\]
where the constant is independent of the \(\lambda_i\)'s and
depends only on \(n\).
\end{lemma}

\begin{remark}
Equivalently, one often writes
\[
  dW
  \;=\;
  \bigl|\Delta(\lambda_1,\dots,\lambda_n)\bigr|\;
  d\Lambda\,dQ,
  \quad\text{where }
  \Delta(\lambda_1,\dots,\lambda_n)
  = \prod_{i<j}(\lambda_j-\lambda_i)
\]
is the \emph{Vandermonde determinant}.
\end{remark}

We prove \Cref{lemma:Jacobian-GOE} in the rest of this subsection.
\medskip

Consider small perturbations of \(\Lambda\) and \(Q\).  Write
\[
  W = Q\,\Lambda\,Q^\top,
  \quad
  \Lambda = \mathrm{diag}(\lambda_1,\dots,\lambda_n).
\]
Let \(\delta W\) be an infinitesimal change in \(W\). We want to see how \(\delta W\) depends on \(\delta\Lambda\) and \(\delta Q\).

\paragraph{Parametrizing \(\delta Q\).}
Since \( Q \in O(n) \), any small variation of \( Q \) can be expressed as
\[ Q \exp(B) \approx Q(I + B), \] where \( B \) is an infinitesimal
skew-symmetric matrix (\( B^\top = -B \)). Indeed,
$\exp(B)$ must be orthogonal, so $\exp(B)^\top \exp(B) = I$.
Thus, we have
\begin{equation*}
	(I+B)^\top(I+B)=I,\qquad \text{or}\qquad B^\top + B = 0.
\end{equation*}
Note that $\exp(B)$ is the matrix exponential of $B$,
which is defined by the usual power series.
Note also that the dimension of \( O(n) \)
is \( \dim(O(n)) = \frac{n(n-1)}{2} \), which matches the dimension of the
space of skew-symmetric matrices.

\paragraph{Computing \(\delta W\).}
Under an infinitesimal change, say,
\[
  Q \mapsto Q\,(I + B),
  \quad
  \Lambda \mapsto \Lambda + \delta\Lambda,
\]
we have
\[
  W
  = Q\ssp\Lambda\ssp Q^\top
  \;\;\Longrightarrow\;\;
  Q^\top \delta W Q
	=
	\delta\Lambda
	+
	B\Lambda-\Lambda B,
\]
to first order in small quantities.
Here we used the orthogonality of $Q$ and
the skew-symmetry of~$B$.

\paragraph{Local structure of the map.}

We see that the map
$W\mapsto(\Lambda,Q)$
in a neighborhood of $(\Lambda,Q)$ determined by
$\delta \Lambda$ and $B$
locally translates by $Q^\top\ssp\delta \Lambda\ssp Q$,
which implies the Lebesgue factor
$d\lambda_1 \ldots d\lambda_n $
in $\delta W$. Indeed, the Lebesgue measure
on $\mathbb{R}^n$ is invariant under orthogonal transformations.

The next terms, the commutator $[B,\Lambda]$, has the form
(recall that $B$ is infinitesimally small and $\Lambda$ is diagonal):
\begin{align*}
B\Lambda-\Lambda B&=
\begin{pmatrix}
	0 & b_{12} & \cdots \\
	-b_{12} & 0 & \cdots \\
	\vdots & \vdots & \ddots
\end{pmatrix}
\begin{pmatrix}
	\lambda_1 & 0 & \cdots \\
	0 & \lambda_2 & \cdots \\
	\vdots & \vdots & \ddots
\end{pmatrix}-
\begin{pmatrix}
	\lambda_1 & 0 & \cdots \\
	0 & \lambda_2 & \cdots \\
	\vdots & \vdots & \ddots
\end{pmatrix}
\begin{pmatrix}
	0 & b_{12} & \cdots \\
	-b_{12} & 0 & \cdots \\
	\vdots & \vdots & \ddots
\end{pmatrix}\\&
=
\begin{pmatrix}
	0 & b_{12}\lambda_2 & \cdots \\
	-b_{12}\lambda_1 & 0 & \cdots \\
	\vdots & \vdots & \ddots
\end{pmatrix}
-
\begin{pmatrix}
	0 & b_{12}\lambda_1 & \cdots \\
	b_{12}\lambda_2 & 0 & \cdots \\
	\vdots & \vdots & \ddots
\end{pmatrix}
\\&=
\begin{pmatrix}
	0 &b_{12}(\lambda_2-\lambda_1) & \cdots \\
	b_{12}(\lambda_1-\lambda_2) & 0 & \cdots \\
	\vdots & \vdots & \ddots
\end{pmatrix}.
\end{align*}
Thus, this action locally means that the 
infinitesimal $b_{ij}$ is multiplied by $\lambda_i-\lambda_j$, 
for all $1\le i<j\le n$.
This is a scalar factor that does not depend on 
the orthogonal component $Q$, but only on the eigenvalues.
Therefore, this factor is the same in 
$Q^\top \ssp \delta W \ssp Q$. 

This completes the proof of 
\Cref{lemma:Jacobian-GOE}.
See also Problem~\ref{prob:Jacobian-GUE} for the GUE Jacobian.

\subsection{Step D. Final Form of the density}
\label{subsec:final-form}

Putting Steps A--C together, we find:
\[
  dW
  \;=\;
	\mathrm{const}\cdot
  \prod_{i<j}|\lambda_i-\lambda_j|\ssp
  d\Lambda
  \,\bigl(\underbrace{\text{Haar measure on }O(n)}_{\text{does not depend on }\lambda_i}\bigr).
\]
Hence, the joint density of \(\{\lambda_1,\dots,\lambda_n\}\) is,
up to normalization depending only on \(n\), equal to
\[
  \prod_{i<j}|\lambda_i-\lambda_j|\;
  \exp\Bigl(-\tfrac{1}{2}\sum_{k=1}^n \lambda_k^2\Bigr).
\]
We leave the computation of the normalization constant in
\Cref{thm:GOE-joint-eigs-detailed} as Problem~\ref{prob:GOE-normalization}.

%%%%%%%%%%%%%%%%%%%%%%%%%%%%%%%%%%%%%%%%%%%%%%%%%%%%%%%%%%%%
\section{Tridiagonal (Householder) Form for real symmetric Matrices}
\label{sec:householder}
%%%%%%%%%%%%%%%%%%%%%%%%%%%%%%%%%%%%%%%%%%%%%%%%%%%%%%%%%%%%

We now give a step-by-step procedure (and proof) of how any real symmetric matrix can be orthogonally transformed into a tridiagonal matrix. This is a standard topic in numerical linear algebra (the “Householder reduction”) but is also central in random matrix theory (especially the Dumitriu–Edelman approach to the Gaussian ensembles).

\subsection{Statement}

\begin{theorem}[real symmetric Tridiagonalization]
\label{thm:tridiagonal}
Any real symmetric matrix \(A\in\mathbb{R}^{n\times n}\) can be represented as
\[
  A = Q^\top\, T\, Q,
  \quad
  \text{where } Q\in O(N)
  \text{ and } T\text{ is real symmetric tridiagonal.}
\]
That is, \(T\) has nonzero entries only on the main diagonal and the first sub- and super-diagonals:
\[
  T = \begin{pmatrix}
         d_1 & \alpha_1 & 0 & \cdots & 0\\
         \alpha_1 & d_2 & \alpha_2 & \cdots & 0\\
         0 & \alpha_2 & d_3 & \ddots & \vdots\\
         \vdots & \vdots & \ddots & \ddots & \alpha_{N-1}\\
         0 & 0 & \cdots & \alpha_{N-1} & d_N
       \end{pmatrix}.
\]
\end{theorem}

\subsection{Householder Reflections: A Detailed Algorithm}

\noindent
\textbf{Householder Reflection (Definition).}
A \emph{Householder reflection} in \(\mathbb{R}^N\) is a matrix \(H\) of the form
\[
  H = I - 2\,\frac{v\,v^\top}{\|v\|^2},
\]
where \(v\in\mathbb{R}^N\) is nonzero. One can check:
\[
  H^\top = H,\quad
  H^2 = I,\quad
  H\text{ is orthogonal, i.e.\ }H^\top H = I.
\]
Geometrically, \(H\) reflects vectors across the hyperplane orthogonal to \(v\).

\medskip
\noindent
\textbf{Goal.}
We want to apply successive Householder reflections to “zero out” all sub-subdiagonal (and super-subdiagonal by symmetry) entries of \(A\), leaving only the main diagonal and the first super-/sub-diagonal possibly nonzero.

\begin{enumerate}[1.]
\item \textbf{Start with } \(A^{(0)}=A\).
\item \textbf{Step \(k=1\).}
   We aim to zero out entries \(A^{(0)}_{2,1},A^{(0)}_{3,1},\ldots,A^{(0)}_{N,1}\), except for one to remain on the first subdiagonal if needed. Specifically, define the vector
   \[
     x
     = (A^{(0)}_{2,1},\,A^{(0)}_{3,1},\,\dots,A^{(0)}_{N,1})^\top
     \;\in\;\mathbb{R}^{N-1}.
   \]
   We want a Householder \(H_1\) such that
   \[
     H_1\,A^{(0)}\,H_1
     = A^{(1)}
   \]
   has zeros in the first column (and row, by symmetry) except possibly \(A^{(1)}_{2,1}\).

   Concretely, embed \(x\) into \(\tilde x \in \mathbb{R}^N\) by placing a \(0\) in the top slot:
   \[
     \tilde x = (\,0,\, A^{(0)}_{2,1},\ldots,A^{(0)}_{N,1}\,)^\top.
   \]
   Choose
   \[
     v = \tilde x \;+\;\alpha e_1
     \;\in\;\mathbb{R}^N,
   \]
   with \(\alpha\) chosen so that
   \(\|v\|\neq 0\) and \((I - 2vv^\top/\|v\|^2)\,\tilde x\) is a scalar multiple of \(e_1\).  A common choice is
   \[
     \alpha = \pm\,\|\tilde x\|,
   \]
   picking a sign that avoids cancellation.  Define
   \[
     H_1 = I - 2 \,\frac{v\,v^\top}{\|v\|^2}.
   \]
   Then \(H_1\) is an orthogonal, symmetric matrix that kills the sub-subdiagonal entries in column 1.
\item \textbf{Step \(k=2,\dots,N-2\).}
   Inductively, we zero out the \((k+2)\)-th to \(N\)-th entries in the \(k\)-th column (and by symmetry, in the \(k\)-th row).  Each step uses a smaller Householder reflection \(H_k\) acting nontrivially in the lower-right \((N-k+1)\times(N-k+1)\) submatrix.  Then set
   \[
     A^{(k)} = H_k\,A^{(k-1)}\,H_k.
   \]
\item \textbf{End result.}
   After \(N-2\) steps, we get \(A^{(N-2)}\), which is tridiagonal, and
   \[
     A^{(N-2)}
     = (H_{N-2}\cdots H_1)\,A\,(H_1\cdots H_{N-2}).
   \]
   Define
   \[
     Q = H_1 \cdots H_{N-2}.
   \]
   Since each \(H_k\) is orthogonal, \(Q\in O(N)\). Moreover,
   \[
     A^{(N-2)} = Q\,A\,Q^\top
   \]
   has the desired tridiagonal form.
\end{enumerate}

\begin{remark}
This procedure is also used in numerical methods for eigenvalue computations: once you reduce to tridiagonal form, one can apply specialized algorithms (like the QR algorithm) more efficiently.
\end{remark}

\begin{proof}[Proof of \Cref{thm:tridiagonal}]
It is essentially just the algorithmic outline above.  Each step is valid because Householder transformations preserve symmetry: if \(B\) is symmetric, then
\[
  (H B H)_{ij}
  = \sum_{r,s} H_{ir} B_{rs} H_{sj}.
\]
But since \(H\) is symmetric itself, \((H B H)\) remains symmetric.  Also, each step zeroes out the sub-subdiagonal entries in the appropriate column and row, thus eventually forcing a tridiagonal shape.  Finally, the product of all Householder reflections used is an orthogonal matrix. This completes the argument.
\end{proof}

%%%%%%%%%%%%%%%%%%%%%%%%%%%%%%%%%%%%%%%%%%%%%%%%%%%%%%%%%%%%
\section{Wigner’s Semicircle Law via Tridiagonalization}
\label{sec:Wigner-SC-detailed}
%%%%%%%%%%%%%%%%%%%%%%%%%%%%%%%%%%%%%%%%%%%%%%%%%%%%%%%%%%%%

We now present a \emph{detailed} outline of how one proves the Wigner semicircle law for the GOE by using its \emph{random tridiagonal model}. This method is due to Dumitriu and Edelman (2002) and is often considered more direct than Wigner’s original moment method.

\subsection{Dumitriu–Edelman Tridiagonal Model}

\begin{theorem}[Tridiagonal Representation of GOE]
\label{thm:DE-model}
Let \(M\) be an \(N\times N\) GOE matrix (real symmetric) with variance chosen so that the off-diagonal entries have variance \(\tfrac12\) and diagonal entries have variance \(1\).  Then there exists an orthogonal matrix \(Q\) such that
\[
   M = Q^\top\,T\,Q,
\]
where \(T\) is a real symmetric tridiagonal matrix of the special form
\[
   T = \begin{pmatrix}
         d_1 & \alpha_1 & 0 & \cdots \\
         \alpha_1 & d_2 & \alpha_2 & \ddots \\
         0 & \alpha_2 & d_3 & \ddots \\
         \vdots & \ddots & \ddots & \ddots
       \end{pmatrix},
\]
and the random variables \(\{d_i,\alpha_j\}\) are mutually independent with
\[
  d_i \sim \mathcal{N}(0,1),
  \quad
  \alpha_j = \sqrt{\frac{\chi^2_{N-j}}{2}},
\]
where \(\chi^2_{\nu}\) is a chi-square distribution with \(\nu\) degrees of freedom, and equivalently \(\sqrt{\tfrac{\chi^2_\nu}{2}}\) is half the norm of a Gaussian vector in \(\mathbb{R}^\nu\).
\end{theorem}

\begin{remark}
- In short, the diagonal entries \(d_i\) are i.i.d.\ \(\mathcal{N}(0,1)\).
- The subdiagonal entries \(\alpha_1,\dots,\alpha_{N-1}\) are independent with each \(\alpha_j\) distributed like \(\sqrt{\tfrac{\chi^2_{N-j}}{2}}\).
- Off-diagonal entries above the first superdiagonal are all zero, so \(T\) has only \(2N-1\) nontrivial entries (the \(N\) diagonal + \((N-1)\) sub-/super-diagonal).
\end{remark}

\begin{proof}[Sketch of Construction]
This is essentially a specialized version of the Householder procedure (\Cref{sec:householder}), carefully arranged so that each step ends up with exactly the distributions described for \(\alpha_j\) and \(d_i\). One uses the fact that a Gaussian matrix is rotationally invariant in a suitable sense, ensuring that each step’s “residual vector” has an isotropic Gaussian distribution.  Then the norm of that vector yields \(\chi^2\) variables.  Full details appear in \cite{DumitriuEdelman2002} or advanced RMT texts.
\end{proof}

Thus, to study the eigenvalues of the GOE matrix \(M\), we can equivalently study the eigenvalues of the (much sparser) tridiagonal matrix \(T\).

\subsection{Characteristic Polynomial and Three-Term Recurrence}

Consider \(p_N(\lambda) = \det(T - \lambda I)\).  Since \(T\) is tridiagonal, one has the well-known three-term recurrence:
\[
  p_0(\lambda) := 1,\quad
  p_1(\lambda) := (d_1 - \lambda),
\]
\[
  p_{k+1}(\lambda)
  \;=\;
  (d_{k+1} - \lambda)\,p_k(\lambda)
  \;-\;\alpha_k^2\,p_{k-1}(\lambda),
  \quad
  (k=1,\dots,N-1).
\]
The roots of \(p_N(\lambda)\) are precisely the eigenvalues \(\lambda_1,\dots,\lambda_N\) of \(T\).

\subsection{Outline of the Semicircle Limit Proof}

We now want to show that the empirical measure
\[
  L_N
  \;=\;
  \frac{1}{N}\sum_{i=1}^N \delta_{\lambda_i}
\]
converges weakly (almost surely) to the semicircle distribution
\[
  \mu_{\mathrm{sc}}(dx)
  = \frac{1}{2\pi} \sqrt{4 - x^2}\,\mathbf{1}_{|x|\le 2}\,dx.
\]
A typical route has these ingredients:

\begin{enumerate}[1.]
\item \textbf{Law of Large Numbers for \(\alpha_j\).}
   Notice that \(\alpha_j^2 = \tfrac{1}{2}\chi^2_{N-j}\) has mean \(\tfrac{N-j}{2}\).  For large \(N\), it is typically of order \(N\).  More precisely, \(\alpha_j \approx \sqrt{\tfrac{N-j}{2}}\) in a probabilistic sense as \(N\to\infty\).

\item \textbf{Scale invariance.}
   One usually rescales \(T\) by \(\sqrt{N}\).  That is, consider \(\tfrac{1}{\sqrt{N}}\,T\).  Its subdiagonal entries become
   \[
     \frac{\alpha_j}{\sqrt{N}}
     \;\approx\;
     \sqrt{\frac{N-j}{2N}}
     \;\approx\;
     \sqrt{\frac{1-j/N}{2}}
     \quad\text{(for large }N\text{)}.
   \]
   Meanwhile, the diagonal entries become \(\tfrac{d_i}{\sqrt{N}}\), which are \(\mathcal{O}(\tfrac{1}{\sqrt{N}})\).  Hence the subdiagonal terms set the main scale for the “bulk” of the spectrum, while the diagonal is negligible in the large \(N\) limit.

\item \textbf{Asymptotic Analysis of Recurrence.}
   A known fact from orthogonal polynomial theory (or from direct PDE-like arguments on the discrete recurrence) is that the location of the roots of \(p_N(\lambda)\) concentrate where the effective continuum limit of the recurrence matches a certain “Stieltjes equation” whose solution is the semicircle density.

   In more elementary terms, one can check that the \emph{moment generating function} or \emph{Stieltjes transform} of the measure \(L_N\) converges to that of \(\mu_{\mathrm{sc}}\).  Alternatively, one can do a direct argument on the polynomials \(p_k(\lambda)\) by bounding their growth and linking it to an integral equation reminiscent of
   \[
     g(z) = \int \frac{1}{x-z}\,d\mu_{\mathrm{sc}}(x),
   \]
   which leads to a quadratic equation solved by the semicircle’s Cauchy transform.

   For details, see \cite{DumitriuEdelman2002} or \cite{TaoTopics}, as the full proof is somewhat technical but completely rigorous.
\end{enumerate}

The net result is that, \emph{with probability 1}, as \(N\to\infty\), the empirical spectral measure of \(\frac{1}{\sqrt{N}}\,M\) (equivalently of \(\frac{1}{\sqrt{N}}\,T\)) converges to the semicircle distribution on \([-2,2]\):
\[
  \mu_{\mathrm{sc}}(dx)
  \;=\;
  \frac{1}{2\pi}\sqrt{4-x^2}\,\mathbf{1}_{|x|\le 2}\,dx.
\]
This is precisely \emph{Wigner’s semicircle law}.

\begin{remark}[Extensions]
A very similar approach works for the Gaussian Unitary Ensemble (\(\beta=2\)), yielding a random \emph{complex Hermitian} tridiagonal (or banded) matrix.  And for \(\beta=4\), there is an analogous construction with quaternionic entries, usually leading to a block-tridiagonal matrix.  All roads lead to the semicircle law for the limiting global spectrum.
\end{remark}


\appendix
\setcounter{section}{2}

\section{Problems (due 2025-02-22)}

\subsection{Invariance of GOE and GUE}
\label{prob:invariance_GOE_GUE}

Show that the distribution of the GOE and GUE is
invariant under, respectively, orthogonal and unitary conjugation.
For GOE, this means that if \(W\)
is a random GOE matrix and \(Q\) is a fixed orthogonal
matrix of order $n$, then the distribution
of \(QWQ^\top\) is the same as the distribution of \(W\).
(Similarly for GUE.)

\medskip
\noindent
Hint: write the joint density of all entries of GOE/GUE (for instance, GOE
is determined by $n(n+1)/2$ real random independent variables)
in a coordinate-free way.


\subsection{Preimage size for spectral decomposition}
\label{prob:GOE-preimage}

Show that for a real symmetric matrix $W$ with distinct eigenvalues,
if $W=Q\Lambda Q^\top$ is its spectral decomposition
where $Q$ is orthogonal and $\Lambda=\mathrm{diag}(\lambda_1,\ldots,\lambda_n)$
is diagonal with $(\lambda_1\ge \cdots\ge \lambda_n)$,
then there are exactly $2^n$ different choices of $Q$
that give the same matrix $W$.

\subsection{Distinct eigenvalues}
\label{prob:distinct-eigenvalues}

Show that under GOE and GUE, almost surely,
all eigenvalues are distinct.

\subsection{Jacobian for GUE}
\label{prob:Jacobian-GUE}

Arguing similarly to 
\Cref{subsec:jacobian},
show that the Jacobian for the spectral decomposition
of a complex Hermitian matrix is proportional to
\begin{equation*}
	\prod_{1\le i<j\le n}|\lambda_i-\lambda_j|^2.
\end{equation*}
In particular, make sure you understand 
where the factor $2$ comes from in the complex case.

\subsection{Normalization for GOE}
\label{prob:GOE-normalization}

Compute the $n$-dimensional integral
\begin{equation*}
	\int_{\lambda_1<\ldots<\lambda_n } \prod_{i<j}(\lambda_i-\lambda_j)
	\ssp
	\exp\Bigl(-\tfrac{1}{2}\sum_{k=1}^n \lambda_k^2\Bigr)\,d\lambda_1\cdots d\lambda_n.=
	\frac{1}{n!}
	\int_{\mathbb{R}^n}
	\prod_{i<j}|\lambda_i-\lambda_j|\ssp
	\exp\Bigl(-\tfrac{1}{2}\sum_{k=1}^n \lambda_k^2\Bigr)\,d\lambda_1\cdots d\lambda_n.
\end{equation*}



\newpage
\section{Eigenvalue Distributions for Classical Ensembles}

We begin by studying eigenvalue distributions for the three fundamental classes of random matrices. These distributions arise from matrices with different symmetry properties and correspond to the real, complex, and quaternionic cases.

\subsection{Matrix Ensembles with Different Symmetries}

Let $X$ be an $N\times N$ matrix. We consider three cases of random matrices with i.i.d. matrix elements:

\begin{enumerate}[a)]
\item \textbf{Real case:} $X_{ij} \sim \mathcal{N}(0,1)$
\item \textbf{Complex case:} $X_{ij} \sim \mathcal{N}(0,1) + i\mathcal{N}(0,1)$
\item \textbf{Quaternion case:} $X_{ij} \sim \mathcal{N}(0,1) + i\mathcal{N}(0,1) + j\mathcal{N}(0,1) + k\mathcal{N}(0,1)$
\end{enumerate}

For each case, we form a self-adjoint matrix:
\[ M = \frac{1}{2}(X + X^*) \]
where $X^*$ denotes the appropriate adjoint. This construction ensures real eigenvalues and proper spectral properties.

\begin{theorem}[Joint Eigenvalue Distribution]
\label{thm:joint_density}
The eigenvalues $\lambda_1 \leq \lambda_2 \leq \cdots \leq \lambda_N$ of the matrix $M$ have joint probability density:
\[ \frac{1}{Z} \prod_{i<j} |\lambda_i-\lambda_j|^\beta \prod_{i=1}^N \exp\left(-\frac{\lambda_i^2}{2}\right) \]
where:
\begin{itemize}
\item $\beta = 1,2,4$ for cases (a), (b), (c) respectively
\item $Z$ is the normalization constant given by:
\[ Z = \frac{(2\pi)^{N/2}}{N!} \prod_{j=1}^{N-1} \frac{\Gamma(1+\beta(j+1)/2)}{\Gamma(1+\beta/2)} \]
\end{itemize}
This density is often called the "multivariate Gaussian" distribution in this context.
\end{theorem}

\subsection{Proof Strategy}

We will prove this theorem for $\beta=1$ (the real case) and outline the modifications needed for other cases. The proof proceeds in three main steps.

\begin{proof}[Step 1: Matrix Density]
The probability density of the matrix $M$ is proportional to:
\[ \exp\left(-\frac{1}{2}\operatorname{Tr}(M^2)\right) \]

Indeed, we can expand the trace:
\[ \operatorname{Tr}(M^2) = \sum_{i,j} |M_{ij}|^2 = \sum_{i=1}^N M_{ii}^2 + 2\sum_{i<j} |M_{ij}|^2 \]
Each element of $M$ is formed from the corresponding elements of $X$ according to the self-adjointness condition.
\end{proof}

\begin{proof}[Step 2: Eigenvalue Transformation]
Using the spectral decomposition $M = ODO^*$ where $D$ is diagonal with eigenvalues $\lambda_i$ and $O$ is orthogonal/unitary/symplectic (depending on $\beta$), we have:
\[ \exp\left(-\frac{1}{2}\operatorname{Tr}(M^2)\right) = \exp\left(-\frac{1}{2}\sum_{i=1}^N \lambda_i^2\right) = \prod_{i=1}^N \exp\left(-\frac{\lambda_i^2}{2}\right) \]
\end{proof}


\begin{proof}[Step 3: Jacobian Calculation]
The key step is computing the Jacobian of the transformation from matrix elements to eigenvalues and eigenvectors. Consider the map:
\[ \Pi: W_N \times \mathcal{G}(N) \to \mathfrak{sl}_N \]
where:
\begin{itemize}
\item $W_N$ is the space of diagonal matrices with ordered eigenvalues
\item $\mathcal{G}(N)$ is $O(N)$, $U(N)$, or $Sp(N)$ depending on $\beta$
\item $\mathfrak{sl}_N$ is the space of self-adjoint matrices
\end{itemize}

This map is given by:
\[ (\lambda, g) \mapsto g\begin{pmatrix}\lambda_1 & & \\ & \ddots & \\ & & \lambda_N\end{pmatrix}g^* \]

Near the identity element of $\mathcal{G}(N)$, we can write:
\[ g = \exp(B) \approx I + B + \frac{B^2}{2} + \cdots \]
where $B$ is skew-symmetric/skew-Hermitian/skew-quaternionic.

The Jacobian computation yields:
\[ \prod_{i<j} |\lambda_i-\lambda_j|^\beta \]
which explains the appearance of this term in the joint density.
\end{proof}

\section{Laguerre/Wishart Ensemble}

Consider a matrix $X$ of size $N \times M$ with $N < M$ having singular value decomposition:
\[ X = U\begin{pmatrix}s_1 & & 0 \\ & \ddots & \\ 0 & & s_N\end{pmatrix}V \]
where $U$ and $V$ are orthogonal/unitary/symplectic matrices of appropriate sizes.

\begin{theorem}[Wishart Distribution]
Let $X$ be an $N\times M$ matrix with i.i.d. Gaussian elements as in Theorem~\ref{thm:joint_density}. Then the eigenvalues $\lambda_i = s_i^2$ of $XX^*$ have joint density proportional to:
\[ \prod_{i<j} (\lambda_i-\lambda_j)^\beta \prod_{i=1}^N \lambda_i^{\frac{\beta}{2}(M-N+1)-1} e^{-\lambda_i/2} \]
This is known as the "multivariate $\Gamma$-distribution."
\end{theorem}

\section{Jacobi/MANOVA/CCA Ensemble}

Consider two rectangular arrays:
\begin{align*}
X &: N\times T \\
Y &: K\times T \qquad N\leq K\leq T
\end{align*}

Define:
\begin{itemize}
\item $P_X$ = projector onto $N$-dimensional subspace spanned by rows of $X$
\item $P_Y$ = projector onto $K$-dimensional subspace spanned by rows of $Y$
\end{itemize}

The squared canonical correlations are $\min(N,K)$ non-zero eigenvalues of $P_XP_Y$.

\begin{theorem}[Canonical Correlations]
Assume $X$ and $Y$ are independent with i.i.d. Gaussian elements. Then the eigenvalues of $P_XP_Y$ have density proportional to:
\[ \prod_{i<j} (\lambda_i-\lambda_j)^\beta \prod_{i=1}^N \lambda_i^{\frac{\beta}{2}(K-N+1)-1} (1-\lambda_i)^{\frac{\beta}{2}(T-K+1)-1} \]
where $0\leq \lambda_i\leq 1$. This is the "multivariate Beta distribution."
\end{theorem}

\section{General Pattern}

A remarkable feature emerges across these classical ensembles. The eigenvalue distributions consistently take the form:
\[ \prod_{i<j} |\lambda_j-\lambda_i|^\beta \prod_{i=1}^N V(\lambda_i) \]
where:
\begin{itemize}
\item The first term represents logarithmic pairwise interaction
\item $V(\lambda)$ is an appropriate potential function
\item $\beta$ represents the symmetry class (1, 2, or 4)
\end{itemize}

This structure appears in various contexts in random matrix theory and is often referred to as a "log-gas" or "$\beta$-ensemble" system.



























\bibliographystyle{alpha}
\bibliography{bib}


\medskip

\textsc{L. Petrov, University of Virginia, Department of Mathematics, 141 Cabell Drive, Kerchof Hall, P.O. Box 400137, Charlottesville, VA 22904, USA}

E-mail: \texttt{lenia.petrov@gmail.com}


\end{document}
