\documentclass{article}
\usepackage{hyperref}
\usepackage[margin=1in]{geometry}

% Include license information
% MIT License (Copyright (c) 2025 Leonid Petrov)

\title{LMathNotes - Hugo-based Mathematical Research Notes Repository}
\author{Leonid Petrov \\ \small{Licensed under MIT License}}
\date{May 12, 2025}

\begin{document}

\maketitle

\texttt{LMathNotes} is a specialized Hugo-based web application designed for organizing, displaying, and navigating mathematical research notes with a focus on LaTeX content. It provides a comprehensive system for managing academic notes with advanced features for confidentiality, history tracking, and content exploration.

\section{Features}

\subsection{Content Organization}
\begin{itemize}
  \item Chronological organization of notes in dated directories
  \item Specialized handling for LaTeX/TeX files with custom viewer
  \item Support for both standard research notes and handwritten notebook collections
  \item Full-text search across all note content, titles, and descriptions
\end{itemize}

\subsection{Navigation \& Display}
\begin{itemize}
  \item \textbf{Homepage View}: Chronological listing of all notes with expandable descriptions
  \item \textbf{Monthly Notes View}: Notes organized by month with consistent display
  \item \textbf{Handwritten Notebooks View}: Special display for handwritten notebook collections
  \item \textbf{History View}: Chain visualization connecting related notes via prev.txt references
  \item \textbf{TeX Viewer}: Specialized viewer for LaTeX files with syntax highlighting
\end{itemize}

\subsection{Specialized Features}

\subsubsection{TeX File Handling}
\begin{itemize}
  \item Custom TeX viewer with syntax highlighting via Prism.js
  \item Line numbers for easy reference
  \item Soft wrap toggle for improved readability
  \item Automatic preamble separation with collapsible display
  \item One-click download option
  \item File metadata display (size, path)
\end{itemize}

\subsubsection{Search Capabilities}
\begin{itemize}
  \item Real-time searching across all content
  \item Case-smart search (uppercase for exact matches, lowercase for case-insensitive)
  \item Visual feedback with dynamic filtering of results
  \item Keyboard shortcut support (Escape to clear/focus)
  \item ``No results'' messaging with clear user feedback
\end{itemize}

\subsubsection{History Tracking}
\begin{itemize}
  \item Visualization of note evolution through connected history chains
  \item Bidirectional navigation between related notes
  \item Recursive history display with circular reference detection
  \item Client-side JavaScript for dynamic history exploration
\end{itemize}

\subsubsection{Confidential Content Protection}
\begin{itemize}
  \item Confidential notes marked via \texttt{conf.txt} files
  \item Content protection with CSS blur effect
  \item Universal keyboard shortcut (Cmd+B / Ctrl+B) to toggle blur state
  \item Consistent protection across all views (main, monthly, handwritten, history)
  \item Selection prevention for blurred content
\end{itemize}

\subsection{Technical Details}
\begin{itemize}
  \item Built on Hugo static site generator
  \item Custom templates for specialized content types
  \item Local storage for preserving UI state (open/closed sections)
  \item Responsive design for all screen sizes
  \item Custom CSS for distinct styling of different content types
\end{itemize}

\section{File Types \& Conventions}

The system uses several special files to manage content and relationships:

\begin{tabular}{|l|l|}
\hline
\textbf{File} & \textbf{Purpose} \\
\hline
\texttt{description.txt} & Defines note title (first line) and description (remaining text) \\
\texttt{prev.txt} & Lists note IDs that precede the current note in history \\
\texttt{conf.txt} & Marks a note as confidential (triggering blur effect) \\
\texttt{*.tex} & LaTeX source files (rendered in custom viewer) \\
\texttt{*.pdf} & Compiled PDF documents (available for direct viewing/download) \\
\hline
\end{tabular}

Note ID refers to the directory name of a note (e.g., \texttt{2025-05-11-blur-test-1}). These IDs are used in \texttt{prev.txt} files to establish relationships between notes and in URL parameters for the history view.

\section{Directory Structure}

\begin{verbatim}
- content/
  - YYYY-MM/
    - YYYY-MM-DD-Note-Title/
      - YYYY-MM-DD-Note-Title.tex  (Main TeX file)
      - YYYY-MM-DD-Note-Title.pdf  (Compiled PDF)
      - description.txt            (Title and description)
      - prev.txt                   (Historical connections)
      - conf.txt                   (Optional, for confidential notes)
      - additional files...        (Images, bibliographies, etc.)
  - Handwritten_Notebooks/
    - YYYY-MM-NN_Notebook_description/
      - description.txt
      - PDF files...
- layouts/
  - Custom templates for different views
- static/
  - css/
  - js/
  - icons/
\end{verbatim}

\section{URL Structure}

\begin{tabular}{|l|l|}
\hline
\textbf{URL Pattern} & \textbf{Purpose} \\
\hline
\texttt{/} & Homepage with chronological listing of all notes \\
\texttt{/YYYY-MM/} & Monthly note listing \\
\texttt{/Handwritten\_Notebooks/} & Handwritten notebooks listing \\
\texttt{/history?note=YYYY-MM-DD-Note-Title} & History view for specific note \\
\texttt{/tex-viewer.html?path=content/path/to/file.tex} & TeX file viewer \\
\hline
\end{tabular}

\section{Setup Instructions}

\begin{enumerate}
  \item Install Hugo (extended version recommended)
  \item Clone this repository
  \item Run \texttt{hugo server} to start the development server
  \item Access the site at \texttt{http://localhost:1313/}
\end{enumerate}

For production deployment:
\begin{enumerate}
  \item Run \texttt{hugo} to build the static site
  \item Deploy the \texttt{public} directory to your web server
\end{enumerate}

\section{Configuration}

Configuration is managed through \texttt{hugo.toml}:

\begin{itemize}
  \item Custom media types for TeX files
  \item Server headers for correct content type handling
  \item Module mounts for serving content directory as static files
  \item Theme configuration for Paperesque base theme
\end{itemize}

\section{Usage Guidelines}

\subsection{Adding New Notes}
\begin{enumerate}
  \item Create a new directory with the naming pattern \texttt{YYYY-MM-DD-Note-Title}
  \item Add your main TeX file with matching name
  \item Create a \texttt{description.txt} with title and description
  \item (Optional) Add \texttt{prev.txt} to link to previous related notes
  \item (Optional) Add \texttt{conf.txt} if the note contains confidential information
\end{enumerate}

\subsection{Navigation}
\begin{itemize}
  \item Use the main page for chronological browsing
  \item Use search for finding specific content
  \item Use history view to track research evolution
\end{itemize}

\subsection{Working with Confidential Content}
\begin{itemize}
  \item Add an empty \texttt{conf.txt} file to any note directory to mark as confidential
  \item Use Cmd+B (Mac) or Ctrl+B (Windows/Linux) to toggle blur effect
  \item Content remains protected until explicitly revealed by the user
\end{itemize}

\section{License}

This project is licensed under the MIT License.

\begin{verbatim}
Copyright (c) 2025 Leonid Petrov

Permission is hereby granted, free of charge, to any person obtaining a copy
of this software and associated documentation files (the "Software"), to deal
in the Software without restriction, including without limitation the rights
to use, copy, modify, merge, publish, distribute, sublicense, and/or sell
copies of the Software, and to permit persons to whom the Software is
furnished to do so, subject to the following conditions:

The above copyright notice and this permission notice shall be included in all
copies or substantial portions of the Software.

THE SOFTWARE IS PROVIDED "AS IS", WITHOUT WARRANTY OF ANY KIND, EXPRESS OR
IMPLIED, INCLUDING BUT NOT LIMITED TO THE WARRANTIES OF MERCHANTABILITY,
FITNESS FOR A PARTICULAR PURPOSE AND NONINFRINGEMENT. IN NO EVENT SHALL THE
AUTHORS OR COPYRIGHT HOLDERS BE LIABLE FOR ANY CLAIM, DAMAGES OR OTHER
LIABILITY, WHETHER IN AN ACTION OF CONTRACT, TORT OR OTHERWISE, ARISING FROM,
OUT OF OR IN CONNECTION WITH THE SOFTWARE OR THE USE OR OTHER DEALINGS IN THE
SOFTWARE.
\end{verbatim}

\vspace{1cm}
\hrulefill

\begin{center}
\textit{LMathNotes - Organize your mathematical research journey}
\end{center}

\end{document}
